
%%%%%%%%%%%%%%%%%%%%%%%%%%%%%%%%%%%%%%%%%%%%%%%%%%%%%%%%%%%%%%%%%%%%%%%%
%%%%%%%%%%%%%%%%%%%%%% Simple LaTeX CV Template %%%%%%%%%%%%%%%%%%%%%%%%
%%%%%%%%%%%%%%%%%%%%%%%%%%%%%%%%%%%%%%%%%%%%%%%%%%%%%%%%%%%%%%%%%%%%%%%%

%%%%%%%%%%%%%%%%%%%%%%%%%%%%%%%%%%%%%%%%%%%%%%%%%%%%%%%%%%%%%%%%%%%%%%%%
%% NOTE: If you find that it says                                     %%
%%                                                                    %%
%%                           1 of ??                                  %%
%%                                                                    %%
%% at the bottom of your first page, this means that the AUX file     %%
%% was not available when you ran LaTeX on this source. Simply RERUN  %%
%% LaTeX to get the ``??'' replaced with the number of the last page  %%
%% of the document. The AUX file will be generated on the first run   %%
%% of LaTeX and used on the second run to fill in all of the          %%
%% references.                                                        %%
%%%%%%%%%%%%%%%%%%%%%%%%%%%%%%%%%%%%%%%%%%%%%%%%%%%%%%%%%%%%%%%%%%%%%%%%

%%%%%%%%%%%%%%%%%%%%%%%%%%%% Document Setup %%%%%%%%%%%%%%%%%%%%%%%%%%%%

% Don't like 10pt? Try 11pt or 12pt
\documentclass[10pt
%							,spanish
%							,draft
							]{article}

\usepackage{calc}
%symbols - the ones you see on the left of the email and of the phone
\usepackage{bbding} 
%Colors/Graphics
\usepackage{color,graphicx}
\usepackage[table,usenames,dvipsnames]{xcolor}
\usepackage{tikz}
\usetikzlibrary{backgrounds}
\usepackage{wrapfig}
% This is a helpful package that puts math inside length specifications
\usepackage{fontspec}
\usepackage{xunicode}
\usepackage{xltxtra}
\usepackage{polyglossia}
\setdefaultlanguage{spanish}
\setotherlanguages{english}
% \usepackage[british,spanish]{babel}
% \usepackage[latin1]{inputenc}
% \usepackage[T1]{fontenc} 	% Better PDF on screen visualisation (a)
% \usepackage{ae,aecompl}
% \usepackage{titlesec}
% Simpler bibsection for CV sections
% (thanks to natbib for inspiration)
\makeatletter
\newlength{\bibhang}
\setlength{\bibhang}{1em}
\newlength{\bibsep}
 {\@listi \global\bibsep\itemsep \global\advance\bibsep by\parsep}
\newenvironment{bibsection}%
        {\vspace{-\baselineskip}\begin{list}{}{%
       \setlength{\leftmargin}{\bibhang}%
       \setlength{\itemindent}{-\leftmargin}%
       \setlength{\itemsep}{\bibsep}%
       \setlength{\parsep}{\z@}%
        \setlength{\partopsep}{0cm}%
        \setlength{\topsep}{0cm}}}
        {\end{list}\vspace{-0.6\baselineskip}}
\makeatother

% Layout: Puts the section titles on left side of page
\reversemarginpar

%
%         PAPER SIZE, PAGE NUMBER, AND DOCUMENT LAYOUT NOTES:
%
% The next \usepackage line changes the layout for CV style section
% headings as marginal notes. It also sets up the paper size as either
% letter or A4. By default, letter was used. If A4 paper is desired,
% comment out the letterpaper lines and uncomment the a4paper lines.
%
% As you can see, the margin widths and section title widths can be
% easily adjusted.
%
% ALSO: Notice that the includefoot option can be commented OUT in order
% to put the PAGE NUMBER *IN* the bottom margin. This will make the
% effective text area larger.
%
% IF YOU WISH TO REMOVE THE ``of LASTPAGE'' next to each page number,
% see the note about the +LP and -LP lines below. Comment out the +LP
% and uncomment the -LP.
%
% IF YOU WISH TO REMOVE PAGE NUMBERS, be sure that the includefoot line 
% is uncommented and ALSO uncomment the \pagestyle{empty} a few lines
% below.
%

%% Use these lines for letter-sized paper
\usepackage[paper=letterpaper,
            %includefoot, % Uncomment to put page number above margin
            marginparwidth=1.2in,     % Length of section titles
            marginparsep=0.5in,       % Space between titles and text
            margin=1in,               % 1 inch margins
            includemp]{geometry}
 
%% Use these lines for A4-sized paper
%\usepackage[paper=a4paper,
%            %includefoot, % Uncomment to put page number above margin
%            marginparwidth=30.5mm,    % Length of section titles
%            marginparsep=1.5mm,       % Space between titles and text
%            margin=25mm,              % 25mm margins
%            includemp]{geometry}

%% More layout: Get rid of indenting throughout entire document
\setlength{\parindent}{0in}
% \setlength{\unitlength}{1in}
%% This gives us fun enumeration environments. compactitem will be nice.
\usepackage{paralist}

%% Reference the last page in the page number
%
% NOTE: comment the +LP line and uncomment the -LP line to have page
%       numbers without the ``of ##'' last page reference)
%
% NOTE: uncomment the \pagestyle{empty} line to get rid of all page
%       numbers (make sure includefoot is commented out above)
%
\usepackage{fancyhdr,lastpage}
\pagestyle{fancy}
%\pagestyle{empty}      % Uncomment this to get rid of page numbers
\fancyhf{}\renewcommand{\headrulewidth}{0pt}
\fancyfootoffset{\marginparsep + \marginparwidth}
\newlength{\footpageshift}
\setlength{\footpageshift}{0.5\textwidth + 0.5\marginparsep +
0.5\marginparwidth-2in} \lfoot{\hspace{\footpageshift}%
       \parbox{4in}{\, \hfill %
                    \arabic{page} de \protect\pageref*{LastPage} % +LP
%                    \arabic{page}                               % -LP
                    \hfill \,}}
% \fancyhf[HL]{\setlength{\unitlength}{1in}
% \begin{tikzpicture}[remember picture, overlay]
% \draw [style=background, xshift=-6.5cm, yshift=2.5cm, very thin] (0,-290mm)
% rectangle (20mm, 0mm);
% \end{tikzpicture}}
% Finally, give us PDF bookmarks
\usepackage{color,hyperref}
\definecolor{darkblue}{rgb}{0.0,0.0,0.3}%{HTML}{F4D455}
\hypersetup{colorlinks,breaklinks,
            linkcolor=darkblue,urlcolor=darkblue,
            anchorcolor=darkblue,citecolor=darkblue}

\definecolor{shade}{HTML}{D4D7FE}	%light blue shade
\definecolor{text1}{HTML}{2b2b2b}		%text is almost black
\definecolor{backgroundColor}{rgb}{0.0,0.0,0.3}%{HTML}{183090}% {HTML}{F4D455}
%%%%%%%%%%%%%%%%%%%%%%%% End Document Setup %%%%%%%%%%%%%%%%%%%%%%%%%%%%
 

%%%%%%%%%%%%%%%%%%%%%%%%%%% Helper Commands %%%%%%%%%%%%%%%%%%%%%%%%%%%%
%Fonts and Tweaks for XeLaTeX
\font\headers="Corbel:letterspace=5" at 20pt
\font\SectionHeaders="Corbel:letterspace=5" at 14pt
%\font\headers="Qlassik Bold:letterspace=5" at 22pt
%\font\SectionHeaders="Qlassik Bold:letterspace=5" at 14pt
\font\Career="Corbel:color=2b2b2b" at 13pt 
\font\Text="Corbel:color=2b2b2b" at 10pt 
\font\TextAlt="Corbel:color=CC3300" at 11pt
\font\TextSC="Corbel:+smcp, color=2b2b2b" at 11pt 
\font\slash="Corbel:+zero, color=2b2b2b" at 20pt
\font\slashAlt="Corbel:+zero, color=CC3300" at 15pt
\font\ContactHeaders="Corbel" at 170 pt
\font\trick="Corbel:color=FFFFFF" at 0.1 pt
\tikzset{background/.style={fill=backgroundColor}}
\tikzset{background grid/.style = {thick, draw  = blue, step = .5cm}}
%Thumbnail for the Portfolio
\setlength\fboxsep{0.1mm}
\newcommand{% 
	\thumbnail}[2]{
		\raggedright{\href{#1}
			{\trick \raisebox{-1pt}{ }
% 				\fbox{
			\includegraphics[keepaspectratio=true,height=20pt]{#2}}
			\raisebox{20pt}{ }}}
% 			}
%Social ICONS 
\newcommand{\icons}[2]{
		{\href{#1}
		{\trick \raisebox{-1pt}{.}
			\raisebox{-4pt}{
			\includegraphics[width=10pt,height=10pt]{#2}}
		\raisebox{10pt}{.}}}
		}
% The title (name) with a horizontal rule under it
%
% Usage: \makeheading{name}
%
% Place at top of document. It should be the first thing.
\newcommand{\makeheading}[1]%%
        {\hspace*{-\marginparsep minus \marginparwidth}%
         \begin{minipage}[t]{\textwidth + \marginparwidth +\marginparsep}
                {\large\slash #1}\\[-0.15\baselineskip]%
                 \rule{\columnwidth}{1pt}%
         \end{minipage}}

% The section headings
%
% Usage: \section{section name}
%
% Follow this section IMMEDIATELY with the first line of the section
% text. Do not put whitespace in between. That is, do this:
%
%       \section{My Information}
%       Here is my information.
%
% and NOT this:
%
%       \section{My Information}
%
%       Here is my information.
%
% Otherwise the top of the section header will not line up with the top
% of the section. Of course, using a single comment character (%) on
% empty lines allows for the function of the first example with the
% readability of the second example.
\renewcommand{\section}[2]%
        {\pagebreak[2]\vspace{1.3\baselineskip}%
         \phantomsection\addcontentsline{toc}{section}{#1}%
         \hspace{0in}%
         \marginpar{ 
         \raggedright\TextSC#1}\Text#2}

% An itemize-style list with lots of space between items
\newenvironment{outerlist}[1][\enskip\textbullet]%
        {\begin{itemize}[#1]}{\end{itemize}%
         \vspace{-.6\baselineskip}}

% An environment IDENTICAL to outerlist that has better pre-list spacing
% when used as the first thing in a \section
\newenvironment{lonelist}[1][\enskip\textbullet]%
        {\vspace{-\baselineskip}\begin{list}{#1}{%
        \setlength{\partopsep}{0pt}%
        \setlength{\topsep}{0pt}}}
        {\end{list}\vspace{-.6\baselineskip}}

% An itemize-style list with little space between items
\newenvironment{innerlist}[1][\enskip\textbullet]%
        {\begin{compactitem}[#1]}{\end{compactitem}}

% An environment IDENTICAL to innerlist that has better pre-list spacing
% when used as the first thing in a \section
\newenvironment{loneinnerlist}[1][\enskip\textbullet]%
        {\vspace{-\baselineskip}\begin{compactitem}[#1]}
        {\end{compactitem}\vspace{-.6\baselineskip}}

% To add some paragraph space between lines.
% This also tells LaTeX to preferably break a page on one of these gaps
% if there is a needed pagebreak nearby.
\newcommand{\blankline}{\quad\pagebreak[2]}
\newcommand{\halfblankline}{\quad\vspace{-0.5\baselineskip}\pagebreak[3]}
\newcommand{\quarterblankline}{\quad\vspace{0.11\baselineskip}\pagebreak[3]}


% Uses hyperref to link DOI
\newcommand\doilink[1]{\href{http://dx.doi.org/#1}{#1}}
\newcommand\doi[1]{doi:\doilink{#1}}
 
\newlength{\rcollength}\setlength{\rcollength}{2.3in}%

%%%%%%%%%%%%%%%%%%%%%%%% End Helper Commands %%%%%%%%%%%%%%%%%%%%%%%%%%%

%%%%%%%%%%%%%%%%%%%%%%%%% Begin CV Document %%%%%%%%%%%%%%%%%%%%%%%%%%%%

\includeonly{espanol, british}

\begin{document} 

% \begin{spanish}
% \makeheading{Ericson Dumar Cepeda L\'opez}

\section{Informaci\'on personal}
%
% NOTE: Mind where the & separators and \\ breaks are in the following
%       table.
%
% ALSO: \rcollength is the width of the right column of the table
%       (adjust it to your liking; default is 1.85in).
%
\begin{minipage}[t]{\textwidth-\rcollength-0.5cm}
Rinc\'{o}n de Santa Ana\newline
Cra 7 Este \# 25 - 59\newline
Ch\'{i}a, Cundinamarca Colombia
\end{minipage}
\begin{minipage}[t]{\rcollength-0.5cm}
\colorbox{shade}{\textcolor{text1}{ 
% \begin{tikzpicture}[overlay, opacity=0.8, color=black, xshift=0.3cm,
% yshift=-40pt] \draw node {\slash 2.\slashAlt 0};%
% \end{tikzpicture}
\begin{tabular}[t]{c|l}%
%\href{http://sistemas.uniandes.edu.co/}%
%     {Departamento de Ingenier�a de Sistemas y Computaci�n} & \\
%\href{http://www.uniandes.edu.co/}{Universidad de los Andes}
\raisebox{-3pt}{\PhoneHandset}
& (313) 200-1812 \\
\raisebox{-3pt}{\Phone} 
& +57(1) 870-8806 \\
\raisebox{-3pt}{\Envelope}
& 
\href{mailto:ericson.cepeda@sukhaweb.com}{ericson.cepeda@sukhaweb.com}
% &	\icons{http://twitter.com/ericsonlopez} {twitter.png} 
% % 	\icons{http://delicious.com/} 		{rss.png}
% 	\icons{http://www.facebook.com/ericson.lopez} 	{facebook.png}
% % 	\icons{http://www.flickr.com/} 		{rss.png} 
% % 	\icons{http://www.last.fm/} 			{rss.png} 
% % 	\icons{http://www.vimeo.com/} 		{rss.png} 
% % 	\icons{http://www.stumbleupon.com/} 	{rss.png}
% % 	\icons{http://www.reddit.com/} 		{rss.png} 
% 	\icons{http://www.linkedin.com/}		{linked.png} 
\end{tabular}
}}%
\end{minipage}

\section{Perfil}
%
Estudiante de pregrado en ingenier\'ia de sistemas con alta motivaci\'on y capaz
de desarrollar aplicaciones de software en diferentes plataformas, llevando a
cabo las etapas de dise\~{n}o correspondientes, en forma cooperativa o
individual. Reconociendo adem\'as los procesos de negocio que pueden influenciar
la implementaci\'on de una soluci\'on.  Esto es complementado con un amplio conocimiento de diversos sistemas operativos.

\section{Educaci\'on}
%
\textbf{Ingenier\'ia de Sistemas y Computaci\'on \hfill{2005 - presente}}
\begin{outerlist}

\item[] \href{http://www.uniandes.edu.co/}{Universidad de los Andes},
Bogot\'a, Cundinamarca Colombia

\end{outerlist}

\section{Experiencia Profesional}
%
\textbf{Desarrollador Web Junior (freelance)} \hfill \textbf{Noviembre 2010 -
presente}
\begin{outerlist}

\item[] \href{http://convertimedia.com/}{Converti Media}, Bogot\'a,
Colombia 
\begin{innerlist}
\item Miner\'ia de datos en bases de datos de Wordpress, y WHMCS para
manejo de \'ordenes, validaci\'on de usuarios, y verificaci\'on de dominios
disponibles en la Web.
\item Programaci\'on en lenguajes como PHP y Javascript (jQuery) de las
p\'aginas web necesarias para la prestaci\'on de un servicio determinado.
\end{innerlist}

\item[] \href{http://irradiadiseno.com/}{Irradia Dise\~no}, Bogot\'a,
Colombia 
\begin{innerlist}
\item Programaci\'on con base en Joomla de distintos sitios web, orientados a
requerimientos espec\'ificos definidos por el cliente, y/o el dise\~nador.
\item Codificaci\'on en lenguajes PHP y Javascript (jQuery) de las
p\'aginas web requeridas para el lanzamiento de un producto.
\item Administraci\'on del DBMS en el servidor del cliente, para adecuar el
comportamiento de las bases de datos, por medio de SQL y/o MySQL, a los
requerimientos funcionales suministrados.
\end{innerlist}

\item[] \href{http://www.sukhaweb.com/}{Sukha SAS.}, Bogot\'a,
Colombia 
\begin{innerlist}
\item Programaci\'on con base en Joomla, y administraci\'on del sitio web de la
empresa.
\item Codificaci\'on en lenguajes PHP, y Javascript (jQuery) de los
requerimientos funcionales, y no funcionales definidos por la organizaci\'on.
\item Administraci\'on del DBMS en el servidor del cliente, y de la
informaci\'on contenida en las bases de datos correspondientes al negocio.
\end{innerlist}
\end{outerlist}

\halfblankline 

\textbf{Desarrollador de sistemas} \hfill \textbf{Marzo 2011 -
presente}
\begin{outerlist}

\item[] \href{http://www.piedradigital.com/}{Piedra Digital}, Bogot\'a,
Colombia
\begin{innerlist}
\item Integrante del grupo de desarrollo de los proyectos Aurora(CakePHP) y
Mediros2(Grails)
\item Programaci\'on con base en Grails, y Java, de sistemas orientados
a requerimientos funcionales espec\'ificos.
\item Integraci\'on de Javascript (jQuery) y Grails, para crear p\'aginas con
contenido din\'amico.
\end{innerlist}

\end{outerlist}

\pagebreak 

\section{Portafolio}
%
Proyectos en curso o terminados

\begin{outerlist}
\item[SUKHA SAS] \href{http://www.sukhaweb.com/}{sukhaweb.com}
\item[A LA CARTA]
\href{http://www.alacartagourmet.com/}{alacartagourmet.com}
% \item[] \thumbnail	{http://www.voiceoverplace.com/}
% 				{voiceov.png}
% 		\thumbnail	{http://www.alacartagourmet.com/final/}
% 				{alacarta.jpg}
% 		\thumbnail	{http://www.sukhaweb.com/}
% 				{sukha}				
\end{outerlist}

\section{Habilidades t\'ecnicas}
%
Experiencia en hardware y software para redes, y tecnolog\'ia de la
informaci\'on

\halfblankline

Programaci\'on: C, C$+$$+$, Java (JUnit, JSF, Servlet, JSP, J2EE), Groovy
(Grails), JavaScript (jQuery), PHP (CakePHP), SQL (MySQL,
PostgreSQL, SQLServer), Python (UNIX shell scripting), Express-MP, Matlab, VHDL,
Git y otros

\halfblankline

Pruebas de software: Documentaci\'on necesaria para registrar las pruebas a
realizar sobre un proyecto en desarrollo. Incluyendo formatos para
documentaci\'on de errores y resultados de pruebas. Lo anterior, con est\'andares de la IEEE

\halfblankline

Tecnolog\'ia de la informaci\'on: Redes (UDP, TCP, ARP, DNS, Dynamic
        routing), Servicio (Apache, Wiki, Wamp Server, phpMyAdmin,
        cPanel, Wordpress, Joomla, dotProject, Planning Tool)

\halfblankline

Aplicaciones de computador: \LaTeX{}, Eclipse, SSH Secure Shell, BiZZdesign
Architect, paquetes comunes de productividad (para Windows, y plataformas Linux)
        , Adobe: Illustrator, Flash Catalyst, Flash Builder, y Dreamweaver

\halfblankline

Dise\~{n}o asistido: Xilinx, SPICE

\halfblankline

Sistemas operativos: Familia Microsoft Windows, Ubuntu, y otras variantes UNIX

\section{Idiomas}
%
Espa\~{n}ol: Nativo. Ingl\'es: Avanzado.

\halfblankline

Ingl\'es: IELTS: 6.0 (Diciembre 2009) Lectura: Excelente. Escritura: Muy buena.
Escucha y habla: Buena

\halfblankline

International House Bristol (The Language Project), Bristol, UK - Curso: General English 20, 1 de Febrero - 7 de Mayo, 2010.
% \newpage

\section{Intereses y logros}
%
\textbf{Investigaci\'on individual}

\begin{outerlist}

\item[] Agent-based modeling, hybrid systems, distributed algorithms, cloud computing, artificial intelligence%
\end{outerlist}

\halfblankline

\textbf{Curso de lectura integral\hfill Enero - Mayo 2005}
\begin{outerlist}

\item[] \href{http://www.tecnicasamericanas.com/}{\textit{T\'ecnicas Americanas
de Estudio}}, Bogot\'a, Cundinamarca Colombia%
\begin{innerlist}
\item Mejoramiento de comprensi\'on y velocidad de lectura
\end{innerlist}
\end{outerlist}

\halfblankline

\textbf{Servicio voluntario\hfill 25 de Mayo - 5 de Junio 2010}
\begin{outerlist}

\item[] \href{http://www.dipa.dhamma.org/}{Dhamma Dipa Meditation Centre},
Hereford, UK%
\begin{innerlist}
\item Miembro del equipo de cocina para 150 estudiantes de meditacion, durante
10 d\'ias
\end{innerlist}
\end{outerlist}

\section{Referencias}
%
Disponibles en caso de ser requeridas.
% \href{mailto:mfonseca@procalculo.com}{\textbf{M\'onica Fonseca Puentes}}
% \begin{outerlist}
% 
% \item[] Directora Help Desk%
%         \hfill \href{http://www.procalculoprosis.com/}{\textbf{Procalculo Prosis S.A.}}
% \begin{innerlist}
% \item Tel: +57(1) 6501550 Ext.3313    
% \item Cel: (300) 5704573  
% \item Cel: (310) 7534086
% \end{innerlist}
% \end{outerlist}
% 
% \halfblankline
% 
% \href{mailto:samafi85@gmail.com}{\textbf{Sandra Madrigal Fierro}}
% \begin{outerlist}
% 
% \item[] Ingeniera en ICBF (Servicio al consumidor)%
%         \hfill \href{https://www.icbf.gov.co/}{\textbf{ICBF}}
% \begin{innerlist}
% \item Cel: (316) 636-2777    
% \end{innerlist}
% \end{outerlist}
% \end{spanish}
\begin{english}
\makeheading{Ericson Cepeda}

\section{Contact Information}
%
% NOTE: Mind where the & separators and \\ breaks are in the following
%       table.
%
% ALSO: \rcollength is the width of the right column of the table
%       (adjust it to your liking; default is 1.85in).
%
\begin{minipage}[t]{\textwidth-\rcollength-0.5cm}
Rinc\'{o}n de Santa Ana\newline
Cra 7 Este \# 25 - 59\newline
Ch\'{i}a, Cundinamarca Colombia
\end{minipage}
\begin{minipage}[t]{\rcollength-0.5cm}
\colorbox{shade}{\textcolor{text1}{
% \begin{tikzpicture}[overlay, opacity=0.8, color=black, xshift=0.3cm,
% yshift=-40pt] \draw node {\slash 2.\slashAlt 0};%
% \end{tikzpicture}
\begin{tabular}[t]{@{}c|l}%
%\href{http://sistemas.uniandes.edu.co/}%
%     {Departamento de Ingenier�a de Sistemas y Computaci�n} & \\
%\href{http://www.uniandes.edu.co/}{Universidad de los Andes}
\raisebox{-3pt}{\PhoneHandset} 
& (313) 200-1812 \\
\raisebox{-3pt}{\Phone} 
& +57(1) 870-8806 \\
\raisebox{-3pt}{\Envelope}
& 
\href{mailto:e-cepeda@uniandes.edu.co}{e-cepeda@uniandes.edu.co}\\
% &	\icons{http://twitter.com/ericsonlopez} {twitter.png}  
% 	\icons{http://delicious.com/} 		{rss.png}
% 	\icons{http://www.facebook.com/ericson.lopez} 	{facebook.png}
% 	\icons{http://www.flickr.com/} 		{rss.png} 
% 	\icons{http://www.last.fm/} 			{rss.png} 
% 	\icons{http://www.vimeo.com/} 		{rss.png} 
% 	\icons{http://www.stumbleupon.com/} 	{rss.png}
% 	\icons{http://www.reddit.com/} 		{rss.png} 
% 	\icons{http://www.linkedin.com/}		{linked.png} 
\end{tabular}
}}%
\end{minipage}


\section{Profile}
%
Undergraduate student of Computer Science, highly motivated, and in capacity of developing software applications on different platforms, implementing different stages in group or individually. Besides, recognising the business processes involved in the implementation of a solution. This is complemented with a wide knowledge of different operative systems.

%\section{Security Clearance}
%
%Department of Defense Top Secret SCI with polygraph (expired: 2002)

% \section{Citizenship}
%
% Colombia

\section{Education}
%
Computer Science, 2005-present
\begin{outerlist}

\item[] \href{http://www.uniandes.edu.co/}{\textbf{Universidad de los Andes}},
Bogot\'{a}, Colombia

\end{outerlist}

\section{Technical Skills}
%
% Hardware and software experience in networking, and information technology
% 
% \halfblankline
% 
Programming: C, C$+$$+$, Java (JUnit, JSF, Servlet, JSP, J2EE), Groovy
(Grails), JavaScript (jQuery), PHP (CakePHP), SQL (MySQL,
PostgreSQL), Python (UNIX shell scripting), Express-MP, Matlab, VHDL, and others

\halfblankline

Information Technology: Networking (UDP, TCP, ARP, DNS, Dynamic
        routing), Service (Apache, Wiki, Wamp Server, phpMyAdmin,
        cPanel, Wordpress, Joomla, dotProject, Planning Tool)

\halfblankline

Computer Applications: \LaTeX{}, Eclipse, SSH Secure Shell, BiZZdesign
Architect, most common productivity packages (for Windows, and Linux platforms)
        , Adobe: Illustrator, Flash Catalyst, Flash Builder, Dreamweaver, and others

\halfblankline

Computer-Aided Design: Xilinx, SPICE

\halfblankline

Operative Systems: Microsoft Windows family,  Ubuntu, and other UNIX variants

\section{Languages}
%
Spanish: Native. English: Advanced

\halfblankline

English: IELTS: 6.0 (December 2009) Reading: Excellent. Writing: Very good. Listening and speaking: Good

\halfblankline

International House Bristol (The Language Project), Bristol, UK - Curso: General English 20, 1st February - 7th May, 2010

\section{Interests and Achivements}
%
\textbf{Individual research}
\begin{outerlist}
\item[] Agent-based modeling, hybrid systems, distributed algorithms, cloud computing, artificial intelligence%
\end{outerlist}

\halfblankline

% \textbf{Non-academic activities}
% 
% \begin{outerlist}
% 
% \item[] I frequently practise sports, such as basketball or swimming. Altough, I spend some time searching information about science advances and technology, as well as, improving my intellect doing some reading on literature, or being autodidact with some informatics thematics%
% \end{outerlist}
% 
% \blankline

\textbf{Reading course}
\begin{outerlist}

\item[] \href{http://www.tecnicasamericanas.com/}{\textit{T\'ecnicas Americanas
de Estudio}}, Bogot\'a, Colombia%
        \\ \textbf{January - May 2005}
\begin{innerlist}
\item Reading comprenhension, and speed improving
\end{innerlist}
\end{outerlist}

\halfblankline

\textbf{Voluntary service}
\begin{outerlist}

\item[] \href{http://www.dipa.dhamma.org/}{\textit{Dhamma Dipa Meditation Centre}},
Hereford, UK%
        \\ \textbf{25th May - 5th June 2010}
\begin{innerlist}
\item Member of the kitchen team for 150 meditation students, during 10 days
\end{innerlist}
\end{outerlist}

\section{References}
%
Available upon request
%\href{mailto:mfonseca@procalculo.com}{\textbf{M�nica Fonseca Puentes}}
%\begin{outerlist}
%
%\item[] \textit{Directora Help Desk}%
%        \hfill \href{http://www.procalculoprosis.com/}{\textbf{Procalculo Prosis S.A.}}
%\begin{innerlist}
%\item \textit{Tel:} +57(1) 6501550 Ext.3313    
%\item \textit{Cel:} (300) 5704573  
%\item \textit{Cel:} (310) 7534086
%\end{innerlist}
%\end{outerlist}
%
%\blankline
%
%\href{mailto:samafi85@gmail.com}{\textbf{Sandra Madrigal Fierro}}
%\begin{outerlist}
%
%\item[] \textit{Ingeniera en ICBF (Servicio al consumidor)}%
%        \hfill \href{https://www.icbf.gov.co/}{\textbf{ICBF}}
%\begin{innerlist}
%\item \textit{Cel:} (316) 636-2777    
%\end{innerlist}
%\end{outerlist}
\end{english}

\end{document}

%%%%%%%%%%%%%%%%%%%%%%%%%% End CV Document %%%%%%%%%%%%%%%%%%%%%%%%%%%%%