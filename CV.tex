
%%%%%%%%%%%%%%%%%%%%%%%%%%%%%%%%%%%%%%%%%%%%%%%%%%%%%%%%%%%%%%%%%%%%%%%%
%%%%%%%%%%%%%%%%%%%%%% Simple LaTeX CV Template %%%%%%%%%%%%%%%%%%%%%%%%
%%%%%%%%%%%%%%%%%%%%%%%%%%%%%%%%%%%%%%%%%%%%%%%%%%%%%%%%%%%%%%%%%%%%%%%%

%%%%%%%%%%%%%%%%%%%%%%%%%%%%%%%%%%%%%%%%%%%%%%%%%%%%%%%%%%%%%%%%%%%%%%%%
%% NOTE: If you find that it says                                     %%
%%                                                                    %%
%%                           1 of ??                                  %%
%%                                                                    %%
%% at the bottom of your first page, this means that the AUX file     %%
%% was not available when you ran LaTeX on this source. Simply RERUN  %%
%% LaTeX to get the ``??'' replaced with the number of the last page  %%
%% of the document. The AUX file will be generated on the first run   %%
%% of LaTeX and used on the second run to fill in all of the          %%
%% references.                                                        %%
%%%%%%%%%%%%%%%%%%%%%%%%%%%%%%%%%%%%%%%%%%%%%%%%%%%%%%%%%%%%%%%%%%%%%%%%

%%%%%%%%%%%%%%%%%%%%%%%%%%%% Document Setup %%%%%%%%%%%%%%%%%%%%%%%%%%%%

% Don't like 10pt? Try 11pt or 12pt
\documentclass[11pt
%							,spanish
%							,draft
							]{article}
 
\usepackage{calc}
%symbols - the ones you see on the left of the email and of the phone
\usepackage{bbding} 
% \usepackage{marvosym} 
%Colors/Graphics
\usepackage{color,graphicx}
\usepackage[table,usenames,dvipsnames]{xcolor}
\usepackage[pscoord]{eso-pic}% The zero point of the coordinate systemis the lower left corner of the page (the default).
\usepackage{tikz}
\usetikzlibrary{backgrounds}
\usepackage{background}
\usepackage{wrapfig}

\usepackage{fontawesome}
% \usepackage{boxedminipage}
% This is a helpful package that puts math inside length specifications
\usepackage{fontspec}
\usepackage{xunicode}
\usepackage{xltxtra}
\usepackage{polyglossia}
\usepackage{setspace}
\setdefaultlanguage{spanish}
\setotherlanguages{english}
% \usepackage[british,spanish]{babel}
% \usepackage[latin1]{inputenc}
% \usepackage[T1]{fontenc} 	% Better PDF on screen visualisation (a)
% \usepackage{ae,aecompl}
% \usepackage{titlesec}
% Simpler bibsection for CV sections
% (thanks to natbib for inspiration)
\makeatletter
\newlength{\bibhang}
\setlength{\bibhang}{1em}
\newlength{\bibsep}
 {\@listi \global\bibsep\itemsep \global\advance\bibsep by\parsep}
\newenvironment{bibsection}%
        {\vspace{-\baselineskip}\begin{list}{}{%
       \setlength{\leftmargin}{\bibhang}%
       \setlength{\itemindent}{-\leftmargin}%
       \setlength{\itemsep}{\bibsep}%
       \setlength{\parsep}{\z@}%
        \setlength{\partopsep}{0cm}%
        \setlength{\topsep}{0cm}}}
        {\end{list}\vspace{-0.6\baselineskip}}
\makeatother

% Layout: Puts the section titles on left side of page
\reversemarginpar

%
%         PAPER SIZE, PAGE NUMBER, AND DOCUMENT LAYOUT NOTES:
%
% The next \usepackage line changes the layout for CV style section
% headings as marginal notes. It also sets up the paper size as either
% letter or A4. By default, letter was used. If A4 paper is desired,
% comment out the letterpaper lines and uncomment the a4paper lines.
%
% As you can see, the margin widths and section title widths can be
% easily adjusted.
%
% ALSO: Notice that the includefoot option can be commented OUT in order
% to put the PAGE NUMBER *IN* the bottom margin. This will make the
% effective text area larger.
%
% IF YOU WISH TO REMOVE THE ``of LASTPAGE'' next to each page number,
% see the note about the +LP and -LP lines below. Comment out the +LP
% and uncomment the -LP.
%
% IF YOU WISH TO REMOVE PAGE NUMBERS, be sure that the includefoot line 
% is uncommented and ALSO uncomment the \pagestyle{empty} a few lines
% below.
%

%% Use these lines for letter-sized paper
\usepackage[paper=letterpaper,
            %includefoot, % Uncomment to put page number above margin
            marginparwidth=0.92in,     % Length of section titles
            marginparsep=0.5in,       % Space between titles and text
            margin=1in,               % 1 inch margins
            includemp]{geometry}
 
%% Use these lines for A4-sized paper
%\usepackage[paper=a4paper,
%            %includefoot, % Uncomment to put page number above margin
%            marginparwidth=30.5mm,    % Length of section titles
%            marginparsep=1.5mm,       % Space between titles and text
%            margin=25mm,              % 25mm margins
%            includemp]{geometry}

%% More layout: Get rid of indenting throughout entire document
\setlength{\parindent}{0in}
% \setlength{\unitlength}{1in}
%% This gives us fun enumeration environments. compactitem will be nice.
\usepackage{paralist}

%% Reference the last page in the page number
%
% NOTE: comment the +LP line and uncomment the -LP line to have page
%       numbers without the ``of ##'' last page reference)
%
% NOTE: uncomment the \pagestyle{empty} line to get rid of all page
%       numbers (make sure includefoot is commented out above)
%
\usepackage{fancyhdr,lastpage}
\pagestyle{fancy}
%\pagestyle{empty}      % Uncomment this to get rid of page numbers
\fancyhf{}\renewcommand{\headrulewidth}{0pt}
\fancyfootoffset{\marginparsep + \marginparwidth}
\newlength{\footpageshift}
\setlength{\footpageshift}{0.5\textwidth + 0.5\marginparsep +
0.5\marginparwidth-2in} \lfoot{\hspace{\footpageshift}%
       \parbox{4in}{\, \hfill %
                    \arabic{page} of \protect\pageref*{LastPage} % +LP
%                    \arabic{page}                               % -LP
                    \hfill \,}}
% \fancyhf[HL]{\setlength{\unitlength}{1in}
% \begin{tikzpicture}[remember picture, overlay]
% \draw [style=background, xshift=-6.5cm, yshift=2.5cm, very thin] (0,-290mm)
% rectangle (20mm, 0mm);
% \end{tikzpicture}}
% Finally, give us PDF bookmarks
\usepackage{color,hyperref}
\definecolor{darkblue}{rgb}{0.0,0.0,0.3}%{HTML}{F4D455}
\definecolor{darkgreen}{HTML}{066769}
\definecolor{namebox}{HTML}{45B6BC}

\definecolor{stripe}{HTML}{E4F4F4}
\definecolor{shade}{HTML}{D4D7FE}	%light blue shade
\definecolor{text1}{HTML}{2b2b2b}		%text is almost black
\definecolor{text2}{HTML}{2D2D2D}		%text is almost black
\definecolor{textitem}{HTML}{5DC8CE}		%text is almost black
\definecolor{subtitle}{HTML}{A1DFE3}		%text is almost black
\definecolor{backgroundColor}{rgb}{0.0,0.0,0.3}%{HTML}{183090}% {HTML}{F4D455}
%%%%%%%%%%%%%%%%%%%%%%%% End Document Setup %%%%%%%%%%%%%%%%%%%%%%%%%%%%
 

%%%%%%%%%%%%%%%%%%%%%%%%%%% Helper Commands %%%%%%%%%%%%%%%%%%%%%%%%%%%%
%Fonts and Tweaks for XeLaTeX


\setmainfont[Color=text2]{Calibri}
\font\headers="Calibri:letterspace=5" at 20pt
\font\SectionHeaders="Calibri:letterspace=5" at 14pt
%\font\headers="Qlassik Bold:letterspace=5" at 22pt
%\font\SectionHeaders="Qlassik Bold:letterspace=5" at 14pt
\font\Career="Calibri:color=2b2b2b" at 13pt 
\font\Text="Calibri:color=2D2D2D" at 11pt 
\font\TextAlt="Calibri:color=CC3300" at 11pt
\font\TextSC="Bebas Neue:+smcp, color=066769, letterspace=4" at 16pt 
\font\slash="Bebas Neue:+zero, color=2b2b2b, letterspace=4" at 35.5pt
\font\slashWhite="Bebas Neue:+zero, color=fffffe, letterspace=4" at 35.5pt
\font\slashGray="Bebas Neue:+zero, color=808080, letterspace=4" at 35.5pt
\font\slashAlt="Corbel:+zero, color=CC3300" at 15pt
\font\ContactHeaders="Corbel" at 170 pt
\font\trick="Corbel:color=FFFFFF" at 0.1 pt
\newfontfamily\FA[Color=darkgreen]{FontAwesome}

\hypersetup{colorlinks,breaklinks,
            linkcolor=darkgreen,urlcolor=darkgreen,
            anchorcolor=darkgreen,citecolor=darkgreen,filecolor=darkgreen}
\makeatletter
\def\HyColor@@@@UseColor#1\@nil{\addfontfeatures{Color=#1}}
\makeatother

\tikzset{background/.style={fill=backgroundColor}}
\tikzset{background grid/.style = {thick, draw  = blue, step = .5cm}}
%Thumbnail for the Portfolio
\setlength\fboxsep{0.1mm}
\newcommand{% 
	\thumbnail}[2]{
		\raggedright{\href{#1}
			{\trick \raisebox{-1pt}{ }
% 				\fbox{
			\includegraphics[keepaspectratio=true,height=20pt]{#2}}
			\raisebox{20pt}{ }}}
% 			}
%Social ICONS 
\newcommand{\icons}[2]{
		{\href{#1}
		{\trick \raisebox{-1pt}{.}
			\raisebox{-4pt}{
			\includegraphics[width=10pt,height=10pt]{#2}}
		\raisebox{10pt}{.}}}
		}
% The title (name) with a horizontal rule under it
%
% Usage: \makeheading{name}
%
% Place at top of document. It should be the first thing.
\newcommand{\makeheading}[2]%%
        {\hspace*{-\marginparsep minus \marginparwidth}%
         \begin{minipage}[l]{\textwidth + \marginparwidth
         +\marginparsep} {\large\slash\setlength{\fboxsep}{2.2mm}
                \hspace*{-2\fboxsep}\colorbox{namebox}{\slashWhite #1}
                \slashGray #2}%
         \end{minipage}}
         
 \newcommand{\placetextbox}[3]{% \placetextbox{<horizontal pos>}{<vertical pos>}{<stuff>}
  \setbox0=\hbox{#3}% Put <stuff> in a box
  \AddToShipoutPictureFG*{% Add <stuff> to current page foreground
    \put(\LenToUnit{#1\paperwidth},\LenToUnit{#2\paperheight}){\vtop{{\null}\makebox[0pt][c]{#3}}}%
  }%
}%
% The section headings
%
% Usage: \section{section name}
%
% Follow this section IMMEDIATELY with the first line of the section
% text. Do not put whitespace in between. That is, do this:
%
%       \section{My Information}
%       Here is my information.
%
% and NOT this:
%
%       \section{My Information}
%
%       Here is my information.
%
% Otherwise the top of the section header will not line up with the top
% of the section. Of course, using a single comment character (%) on
% empty lines allows for the function of the first example with the
% readability of the second example.
\renewcommand{\section}[2]%
        {\pagebreak[2]\vspace{1.3\baselineskip}%
         \phantomsection\addcontentsline{toc}{section}{#1}%
         \hspace{0in}%
         \marginpar{ 
         \vspace{-8pt}\raggedright\setstretch{1.4}\TextSC#1}\Text#2}

% An itemize-style list with lots of space between items
\newenvironment{outerlist}[1][\textbullet]%
        {\begin{itemize}[#1]}{\end{itemize}%
         \vspace{-.45\baselineskip}}

% An environment IDENTICAL to outerlist that has better pre-list spacing
% when used as the first thing in a \section
\newenvironment{lonelist}[1][\enskip\textbullet]%
        {\vspace{-\baselineskip}\begin{list}{#1}{%
        \setlength{\partopsep}{0pt}%
        \setlength{\topsep}{0pt}}}
        {\end{list}\vspace{-.6\baselineskip}}

% An itemize-style list with little space between items
\newenvironment{innerlist}[1][\textbullet]%
        {\begin{compactitem}[#1]}{\end{compactitem}}

% An environment IDENTICAL to innerlist that has better pre-list spacing
% when used as the first thing in a \section
\newenvironment{loneinnerlist}[1][\textbullet]%
        {\vspace{-\baselineskip}\begin{compactitem}[#1]}
        {\end{compactitem}\vspace{-.6\baselineskip}}

% To add some paragraph space between lines.
% This also tells LaTeX to preferably break a page on one of these gaps
% if there is a needed pagebreak nearby.
\newcommand{\blankline}{\quad\pagebreak[2]}
\newcommand{\halfblankline}{\quad\vspace{-0.5\baselineskip}\pagebreak[3]}
\newcommand{\quarterblankline}{\quad\vspace{0.1\baselineskip}\pagebreak[3]}
\newcommand{\smallblankline}{\vspace{0.3\baselineskip}\pagebreak[3]}

% Uses hyperref to link DOI
\newcommand\doilink[1]{\href{http://dx.doi.org/#1}{#1}}
\newcommand\doi[1]{doi:\doilink{#1}}
 
\newlength{\rcollength}\setlength{\rcollength}{2.3in}%

\SetBgScale{1}
\SetBgAngle{0}
\SetBgColor{stripe}
\SetBgContents{\rule{1.5in}{\paperheight}}
\SetBgHshift{-.5\paperwidth+1.577in}

%%%%%%%%%%%%%%%%%%%%%%%% End Helper Commands %%%%%%%%%%%%%%%%%%%%%%%%%%%

%%%%%%%%%%%%%%%%%%%%%%%%% Begin CV Document %%%%%%%%%%%%%%%%%%%%%%%%%%%%

\includeonly{espanol, british, espanol_web, espanol_data}

\begin{document} 

\begin{spanish}
\makeheading{Ericson}{Cepeda}{
\placetextbox{0.75}{.97}{
%
		\begin{tikzpicture}[remember picture,overlay]
		\draw[dashed, color=namebox, thin, double, double distance=0pt] (0pt,.5cm)
		-- (0pt,-3cm);
		\end{tikzpicture}
		\hspace*{0pt}\makebox[5cm][c]{%
		\def\arraystretch{1.1}%
		\begin{tabular}[t]{cl}%
%\href{http://sistemas.uniandes.edu.co/}%
%     {Departamento de Ingenier\'ia de Sistemas y Computaci\'on} & \\
%\href{http://www.uniandes.edu.co/}{Universidad de los Andes}
% \raisebox{-1pt}{\FA \faMobile} 
% & \small +57 3005574311 \\
\raisebox{-1pt}{\FA \faGlobe}
& 
\href{http://www.picorb.com}{picorb.com} \\
\raisebox{-1pt}{\FA \faEnvelopeSquare}
& 
\href{mailto:ericson@picorb.com}{ericson@picorb.com} \\
\raisebox{-1pt}{\FA \faLinkedinSquare}
& 
\href{https://co.linkedin.com/in/ericsoncepeda}{ericsoncepeda} \\
\raisebox{-1pt}{\FA \faGithub}
& 
\href{https://github.com/ericson-cepeda}{ericson-cepeda} \\
\raisebox{-1pt}{\FA \faBitbucket}
& 
\href{https://bitbucket.org/ericson_cepeda}{ericson\_cepeda} \\
\raisebox{-1pt}{\FA \faPhone} 
&
\small +1 857-6000106 %\\
%\raisebox{-1pt}{\FA \faMapMarker}
%& 
%\small Cra 7 Este 25-59, Chía CO-CUN
% &	\icons{http://twitter.com/ericsonlopez} {twitter.png} 
% % 	\icons{http://delicious.com/} 		{rss.png}
% 	\icons{http://www.facebook.com/ericson.lopez} 	{facebook.png}
% % 	\icons{http://www.flickr.com/} 		{rss.png} 
% % 	\icons{http://www.last.fm/} 			{rss.png} 
% % 	\icons{http://www.vimeo.com/} 		{rss.png} 
% % 	\icons{http://www.stumbleupon.com/} 	{rss.png}
% % 	\icons{http://www.reddit.com/} 		{rss.png} 
% 	\icons{http://www.linkedin.com/}		{linked.png} 
\end{tabular}
}}%
}%

\vspace{0.7cm}

\section{Perfil}
%
%Ingeniero \textit{full stack} de sistemas y computaci\'on, altamente
%motivado y capaz de implementar aplicaciones de software en diferentes
%plataformas, llevando a cabo etapas en forma cooperativa o
%individual, haciendo uso de metodolog\'ias \'agiles. Reconociendo adem\'as, los
%procesos de negocio que pueden influenciar la implementaci\'on de una
% soluci\'on.  Esto es complementado con un amplio conocimiento de diversos sistemas operativos.
% Ingeniero \textit{full stack} de sistemas y computación, curioso acerca de
% nuevos avances y cómo aprovecharlos apuntando a construir un mejor lugar para coexistir y disfrutar. 
% Autodidacta, pero ávido de conocimiento cuando mentores y visiones mixtas se encuentran alrededor. 
% Consciente de la increíble habilidad de los humanos para aprender y con grandes
% deseos de compartir conocimientos útiles adquirido.
% Despierto a este mundo ágil y acelerado en el que vivimos sabiendo que puede regirse por procesos y reglas, 
% pero también el caos que trae consigo nuevas dimensiones a la realidad.
Ingeniero de software con experiencia corporativa, independiente y de
emprendimiento. Liderazgo y actividades auto-didactas
han servido para desplegar una infraestructura con excelente disponibilidad en
Twnel. Ingeniar soluciones para distintos clientes y la mezcla
de diversos perfiles en Bazaarvoice han desarrollado habilidades para
innovar en sus prácticas actuales. Como independiente, curioso por nuevos
avances y sus ventajas para así llevar a cabo múltiples proyectos en PicOrb. Roles que
involucren liderazgo y gestión son enriquecedores y, un paso adelante.

\vspace{3.5mm}
\textbf{Español} Nativo. \textbf{Inglés} Avanzado. \textbf{Mandarín}
Principiante.

\section{Educaci\'on}
%
\href{http://www.topuniversities.com/universities/universidad-de-los-andes}{\textbf{Universidad de los Andes}},
Bogot\'{a}, Colombia
\begin{outerlist}
\item[\FA \faAngleDoubleRight] \textbf{Ingenier\'ia de Sistemas y Computaci\'on}
\hfill \textbf{2014}
\end{outerlist}
     \begin{innerlist}
     	\item Top 10 universidades LatAm.
     \end{innerlist}

\section{Experiencia}
%
\href{http://www.bazaarvoice.com/}{\textbf{Bazaarvoice}}, (remoto)
\textit{Austin, TX, USA. La mayor red de datos del consumidor.}

\begin{outerlist}
\item[\FA \faAngleDoubleRight] \textbf{Ingeniero de Datos -
Full Stack / Front-end}
\hfill
\textbf{Mar 2016 - Presente}
\end{outerlist}

\begin{innerlist}
\item Innovar con el uso de plantillas y caché en la implementación de sitios
\textit{web} para el muestreo de métricas operacionales, garantizando
mínima latencia (-3 seg.).
\item Estandarizar metodologías de automatización para integración y despliegue
continuos, remplazando procedimientos manuales en 2 equipos de trabajo.
\item \textbf{Tecnología:} PHP, JS, Stylus, Python, Spark, MySQL,
Jenkins, AWS, Docker.
\end{innerlist}

\quarterblankline

\href{http://www.twnel.com/}{\textbf{Twnel Inc.}}, (remoto) \textit{Boston, MA,
USA.
Startup.
Pionera en mensajería empresarial.}

\begin{outerlist}
\item[\FA \faAngleDoubleRight] \textbf{Ingeniero de Infraestructura -
Full Stack / Front-end}
\hfill
\textbf{Nov 2014 - Presente}
\end{outerlist}

\begin{innerlist}
\item Alcanzar 100\% en disponibilidad para el cliente
\textit{web} al usar S3 y CloudFront sin intervención humana.
\item Reducir +50\% en costos operacionales y de infraestructura creando
procesos de despliegue continuo con \textit{zero-downtime} en producción.
\item \textbf{Tecnología:} AWS, GCP, Kubernetes, NodeJS,
ReactJS, Python, Ansible, CircleCI. %, CoreOS
\end{innerlist}

\quarterblankline

 
\href{http://www.picorb.com/}{\textbf{PicOrb}}, \textit{Bogotá, Colombia.
Agencia digital. Freelance desde 2010.}
\begin{outerlist}
\item[\FA \faAngleDoubleRight] \textbf{Ingeniero de Software -
Full Stack / Front-end}
\hfill
\textbf{Ene 2013 - Presente}
\end{outerlist}
    \begin{innerlist}
\item Ingeniar el sitio oficial de la compa\~n\'ia.
\item Liderar un equipo din\'amico en el desarrollo de
diferentes proyectos \textit{web} reutilizando 100\% de los componentes
implementados en AngularJS y Django para el sitio oficial de la agencia.
\item Integrar AWS S3, Heroku y Codeship garantizando \textit{Continuous
Integration / Deployment} con \textit{zero-downtime} en producción.
\item \textbf{Tecnología:} Django, AngularJS, AWS, Heroku, Codeship.
    \end{innerlist}
    
\quarterblankline

\href{http://alertlogic.com/}{\textbf{Alert Logic}}, (remoto) \textit{Houston,
TX, USA.
Proveedor de seguridad como servicio.}
\begin{outerlist}
\item[\FA \faAngleDoubleRight] \textbf{Ingeniero de Software de Pruebas} \hfill
\textbf{Mar 2013 - Mar 2016}
\end{outerlist}

    \begin{innerlist}
\item Desarrollar nuevas funcionalidades para WSM en cuanto a
\textit{health checking} y verificaci\'on de paquetes.
\item Crear soluciones para la integraci\'on de herramientas usadas
por 3 equipos y reducir el tiempo para la definici\'on de
documentaci\'on de pruebas en un 65\%.
\item \textbf{Tecnología:} PHP (Selenium Framework), Python, Perl, Erlang,
Ruby, AWS.
    \end{innerlist}

%\quarterblankline

\section{Experiencia Previa}
%
\textbf{Enteract}, (remoto)
\textit{Melbourne, Australia. Líder en mercadeo digital.}
%which gives balance between strategic solutions and their implementation. 

\begin{outerlist}
\item[\FA \faAngleDoubleRight] \textbf{Desarrollador Web} \hfill
\textbf{Julio 2012 - Febrero 2013}
\end{outerlist}
% 
    \begin{innerlist}
%\item Accomplished the production and release of
%two projects in a rapid development team: Aurora and
%\href{http://demo.100grados.co:8080/desempeno100/}{\textbf{Mediros2}}.
% \item Integrated backend libraries for data mining.
\item \textbf{Tecnología:} CakePHP, jQuery.
    \end{innerlist}

\quarterblankline

\textbf{Piedra Digital},
\textit{Bogot\'a, Colombia. Consultoría y desarrollo de software.}
%which gives balance between strategic solutions and their implementation. 

\begin{outerlist}
\item[\FA \faAngleDoubleRight] \textbf{Analista y Desarrollador de Sistemas}
\hfill
\textbf{Marzo 2011 - Junio 2012}
\end{outerlist}
% 
    \begin{innerlist}
%\item Accomplished the production and release of
%two projects in a rapid development team: Aurora and
%\href{http://demo.100grados.co:8080/desempeno100/}{\textbf{Mediros2}}.
% \item Integrated backend libraries for data mining.
\item \textbf{Tecnología:} CakePHP, jQuery, SQL Server, PostgreSQL, Grails.
    \end{innerlist}

\quarterblankline

\textbf{Sukha SAS.}, \textit{Bogot\'a, Colombia. Comercio de productos
orgánicos.}
% Multipurpose company:
% international product trading, software development, home-automation and organic
% products fair-trading.

\begin{outerlist}
\item[\FA \faAngleDoubleRight] \textbf{Ingeniero de Software (co-fundador)}
\hfill \textbf{Enero 2011 - Octubre 2012}
\end{outerlist}

    \begin{innerlist}
% \item Built the company main website seeking for creative,
%minimalistic and attractive ways to show the company profile.
\item \textbf{Tecnología:} CakePHP, JQuery, MySQL DBMS, CPanel.
    \end{innerlist}
%\blankline
% \href{http://convertimedia.com/}{\textbf{Converti Media}}, Bogot\'a, Colombia.
% Empresa de consultor\'ia y desarrollo de estrategias de comunicacion en l\'inea.
% \begin{outerlist}
% \item[] \textbf{Desarrollador Junior} \hfill \textbf{Noviembre 2010 - Mayo 2011}
%     \begin{innerlist}
%     	\item Dise\~no e implementaci\'on de algoritmos necesarios para la
%     	integraci\'on de las funcionalidades de WHMCS con una aplica\'on online
%     	Wordpress.
% 		\item Miner\'ia de datos (MySQL), Desarrollo en WHMCS y Wordpress.
% 		\item PHP, Javascript (jQuery) y servicios web.
%     \end{innerlist}
% \end{outerlist}
% 
% \quarterblankline

% \href{http://irradiadiseno.com/}{\textbf{Irradia Dise\~no}}, Bogot\'a, Colombia.
% Grupos de dise\~nadores con enfoque en las necesidades de las PYMEs en Colombia.
% \begin{outerlist}
% \item[] \textbf{Desarrollador Web (freelance)} \hfill \textbf{Junio 2011 - Julio
% 2011}
%     \begin{innerlist}
% \item Reconstrucci\'on de las funcionalidades b\'asicas de
% \textbf{A La Carta}, habilitando el 100\% de los clientes para
% hacer \'ordenes en l\'inea.
% \item PHP (Joomla) y Javascript (jQuery).
% \item Administraci\'on de DBMS MySQL.
%     \end{innerlist}
% \end{outerlist}
% 
% \quarterblankline

% \textbf{Sukha SAS.}, Bogot\'a, Colombia.
% Compa\~n\'ia multiprop\'osito que integra: Comercio internacional de productos,
% desarrollo software, automatizaci\'on de hogares y comercio de productos
% org\'anicos.
% \begin{outerlist}
% \item[] \textbf{CTO (co-founder)} \hfill \textbf{Febrero 2011 -
% October 2012}
%     \begin{innerlist}
% \item Construcci\'on del sitio web principal de la compa\~n\'ia, en busca de
% formas creativas y minimalistas para mostrar el perfil de la empresa.
% Administraci\'on de la infrestructura tecnol\'ogica.
% \item PHP (CakePHP) y Javascript (jQuery).
% \item Administraci\'on DBMS MySQL.
%     \end{innerlist}
% \end{outerlist}
% 
% \quarterblankline
% 
% \href{http://www.piedradigital.com/}{\textbf{Piedra Digital}}, Bogot\'a,
% Colombia. Compa\~n\'ia de desarrollo software con enfoque en el equilibrio entre
% soluciones estrat\'egicas y su implementaci\'on.
% \begin{outerlist}
% \item[] \textbf{Systems analyst and developer} \hfill \textbf{Marzo 2011 - Junio
% 2012}
%     \begin{innerlist}
% \item Conclusi\'on de la producci\'on y lanzamiento de dos proyectos: Aurora
% (CakePHP) y
% \href{http://demo.100grados.co:8080/desempeno100/}{\textbf{Mediros2}} (Grails).
% \item Java (core, Grails) y Javascript (jQuery).
% \item Miner\'ia de datos con SQL Server y PostgreSQL.
%     \end{innerlist}
% \end{outerlist}
% 
% \quarterblankline
% 
% \href{http://sqbluesky.com/}{\textbf{SQBlueSky Inc.}}, Dallas, TX USA. 
% Compa\~n\'ia financiera enfocada en ofrecer y construir herramientas \'utiles
% para la investigaci\'on de inversi\'on.
% \begin{outerlist}
% \item[] \textbf{Systems analyst and developer} \hfill \textbf{Mayo 2012 -
% Octubre 2012}
%     \begin{innerlist}
% \item Integraci\'on de la aplicaci\'on web: \textbf{Insider Edge}
% y la nueva implementaci\'on CouchDB, para un servicio r\'apido y m\'as
% confiable.
% \item PHP (Zend Framework) y Javascript (jQuery).
% \item Miner\'ia de datos con CouchDB y MySQL.
%     \end{innerlist}
% \end{outerlist}
% 
% \quarterblankline
% 
% \href{http://www.enteract.com.au/}{\textbf{Enteract}}, Melbourne, Australia.
% Empresa l\'ider en mercadeo. Produce un amplio rango de servicios de desarrollo
% web, mercadeo online y eventos de comercio.
% \begin{outerlist}
% \item[] \textbf{Web developer (freelance)} \hfill \textbf{Julio 2012 - Febrero
% 2013}
%     \begin{innerlist}
% \item Coordinaci\'on de la actualizaci\'on del sitio para distribuci\'on de
% tajetas virtuales: 
% \href{https://my.ekarda.com/}{\textbf{Ekarda}} hacia tecnolog\'ias m\'as
% recientes, incrementando la compatibilidad con nuevos requerimientos.
% \item PHP (CakePHP) y Javascript (jQuery).
% \item An\'alisis de c\'odigo con REGEX avanzado para la implementaci\'on de
% i18n en reemplazo de texto plano.
%     \end{innerlist}
% \end{outerlist}
% 
% \quarterblankline


% \pagebreak 

% \section{Portafolio}
%
% Independiente:
% 
% \begin{innerlist}
% \item \href{http://www.picorb.com/}{\textbf{PicOrb}}
% \item \href{http://www.cafetosoftware.com/}{\textbf{Cafeto Software}}
% \item \href{http://www.giant-turkey.com/}{\textbf{Giant Turkey}}
% \item \href{http://www.allbikers.net/}{\textbf{All Bikers}}
% \item \href{https://my.ekarda.com/}{\textbf{Ekarda}}
% \item \href{http://demo.100grados.co:8080/desempeno100/}{\textbf{Mediros2}}
% \item \href{http://www.sukhaweb.com/}{\textbf{Sukha SAS}}
% \item \href{http://www.alacartagourmet.com/}{\textbf{A La Carta}}
% \end{innerlist}
% \begin{outerlist}
% \item[SUKHA SAS] \href{http://www.sukhaweb.com/}{sukhaweb.com}
% \item[A LA CARTA]
% \href{http://www.alacartagourmet.com/}{alacartagourmet.com}
% \item[] \thumbnail	{http://www.voiceoverplace.com/}
% 				{voiceov.png}
% 		\thumbnail	{http://www.alacartagourmet.com/final/}
% 				{alacarta.jpg}
% 		\thumbnail	{http://www.sukhaweb.com/}
% 				{sukha}				
% \end{outerlist}
% \quarterblankline

% Contractual:

% \begin{innerlist}
% \item \href{http://www.picorb.com/}{\textbf{PicOrb}}
% \item \href{http://www.cafetosoftware.com/}{\textbf{Cafeto Software}}
% \item \href{http://www.giant-turkey.com/}{\textbf{Giant Turkey}}
% \item \href{http://www.allbikers.net/}{\textbf{All Bikers}}
% \item \href{https://my.ekarda.com/}{\textbf{Ekarda}}
% \item \href{http://demo.100grados.co:8080/desempeno100/}{\textbf{Mediros2}}
% \item \href{http://www.sukhaweb.com/}{\textbf{Sukha SAS}}
% \item \href{http://www.alacartagourmet.com/}{\textbf{A La Carta}}
% \end{innerlist}

% \newpage

\section{Habilidades t\'ecnicas}
%
\textbf{Programaci\'on:}

    \begin{innerlist}
\item Avanzado: Core Java, Python (Django, Selenium), JavaScript
(NodeJS, ReactJS, AngularJS), SQL (MySQL, PostgreSQL,
SQL Server), NoSQL (MongoDB, CouchDB, Neo4j, DynamoDB), bash (UNIX shell
scripting).
\item Intermedio: Groovy (Grails), PHP
(CakePHP, Zend, Wordpress, Joomla), Perl, Ruby (RoR), 
Java (JUnit, J2EE), \LaTeX{}.
\item Aprendiz: C$+$$+$, C\#, Matlab, VHDL, Hadoop, Clojure,
Erlang, Go.
\item Principiante: Objective C, Express-MP.
    \end{innerlist}

\halfblankline

\textbf{Sistemas operativos:}
    \begin{innerlist}
\item Ubuntu, CoreOS, CentOS y otras variantes UNIX.
\item Microsoft Windows.
    \end{innerlist}
    
\halfblankline

\textbf{Tecnolog\'ia de la informaci\'on:} 
    \begin{innerlist}
\item Networking: Bind, SkyDNS, Consul, Swarm.
\item Servicio: NodeJS, Tornado, Nginx, Apache, Unicorn.
\item Integraci\'on y entrega continua: Jenkins, Codeship, CircleCI,
Fabric, Ansible, Chef, GitLab.
\item Computaci\'on en la nube: Heroku, AWS, Rackspace, Google Cloud Platform.
\item Micro servicios y virtualización: Kubernetes, Docker, Vagrant.
    \end{innerlist}

% \halfblankline
% 
% \textbf{Aplicaciones:} 
%     \begin{innerlist}
% \item Eclipse, SSH Secure Shell, BiZZdesign
% Architect, Adobe: Illustrator, Flash Catalyst, Flash Builder, Dreamweaver, and
% others.
% \item Paquetes y utilidades para plataformas Windows y Linux.
%     \end{innerlist}
% 
% 
% \textbf{Dise\~no asistido:} 
%     \begin{innerlist}
% \item Xilinx, SPICE.
%     \end{innerlist}
% 
% 
% \halfblankline
% 
% \section{Idiomas}
% %
% Espa\~{n}ol: Nativo. Ingl\'es: Avanzado.
% 
% \halfblankline
% 
% Ingl\'es: Lectura: Excelente. Escritura: Muy buena.
% Escucha y habla: Muy buena.
% 
% \halfblankline
% 
% International House Bristol (The Language Project), Bristol, UK - Curso: General English 20, 1 de Febrero - 7 de Mayo, 2010.
% \newpage

\section{Servicio Voluntario}
%
\href{http://www.dipa.dhamma.org/}{\textbf{Dhamma
Dipa Meditation Centre}}, \textit{Hereford, UK}
%
\begin{outerlist}
\item[\FA \faAngleDoubleRight] \textbf{Voluntario}%
        \hfill \textbf{May - Jun 2010}
\end{outerlist}

\begin{innerlist}
\item Asistir 10 días como parte del equipo de cocina para +150
meditadores Vipassana.
\end{innerlist}


% \begin{outerlist}
% \item[] Ha crecido un sentimiento de humanidad y responsabilidad social gracias
% a la meditaci\'on. El objetivo principal es ayudar a las personas con amor
% incondicional y al mundo a ser un mejor lugar para vivir. Existen lugares donde se puede encontrar la motivaci\'on para lograr 
% dicho objectivo con la tecnolog\'ia, por lo que existe ese impulso
% para ser tecnol\'ogicamente ``verde''.
% %
% \end{outerlist}
% 
% \halfblankline
% 
% \textbf{Curso de lectura integral\hfill Enero - Mayo 2005}
% \begin{outerlist}
% 
% \item[] \href{http://www.tecnicasamericanas.com/}{\textit{T\'ecnicas Americanas
% de Estudio}}, Bogot\'a, Cundinamarca Colombia%
% \begin{innerlist}
% \item Mejoramiento de comprensi\'on y velocidad de lectura.
% \end{innerlist}
% \end{outerlist}
% 
% \halfblankline
% 
% \textbf{Servicio voluntario\hfill 25 de Mayo - 5 de Junio 2010}
% \begin{outerlist}
% 
% \item[] \href{http://www.dipa.dhamma.org/}{Dhamma Dipa Meditation Centre},
% Hereford, UK%
% \begin{innerlist}
% \item Miembro del equipo de cocina para 150 estudiantes de meditacion, durante
% 10 d\'ias.
% \end{innerlist}
% \end{outerlist}

% \section{Referencias}
%
% Disponibles en caso de ser requeridas.
% \href{mailto:mfonseca@procalculo.com}{\textbf{M\'onica Fonseca Puentes}}
% \begin{outerlist}
% 
% \item[] Directora Help Desk%
%         \hfill \href{http://www.procalculoprosis.com/}{\textbf{Procalculo Prosis S.A.}}
% \begin{innerlist}
% \item Tel: +57(1) 6501550 Ext.3313    
% \item Cel: (300) 5704573  
% \item Cel: (310) 7534086
% \end{innerlist}
% \end{outerlist}
% 
% \halfblankline
% 
% \href{mailto:samafi85@gmail.com}{\textbf{Sandra Madrigal Fierro}}
% \begin{outerlist}
% 
% \item[] Ingeniera en ICBF (Servicio al consumidor)%
%         \hfill \href{https://www.icbf.gov.co/}{\textbf{ICBF}}
% \begin{innerlist}
% \item Cel: (316) 636-2777    
% \end{innerlist}
% \end{outerlist}
\end{spanish}

% \begin{spanish}
% \makeheading{Ericson Dumar Cepeda L\'opez}

\section{Informaci\'on personal}
%
% NOTE: Mind where the & separators and \\ breaks are in the following
%       table.
%
% ALSO: \rcollength is the width of the right column of the table
%       (adjust it to your liking; default is 1.85in).
%
\begin{minipage}[t]{\textwidth-\rcollength-0.5cm}
Rinc\'{o}n de Santa Ana\newline
Cra 7 Este \# 25 - 59\newline
Ch\'{i}a, Cundinamarca Colombia
\end{minipage}
\begin{minipage}[t]{\rcollength-0.5cm}
\colorbox{shade}{\textcolor{text1}{ 
% \begin{tikzpicture}[overlay, opacity=0.8, color=black, xshift=0.3cm,
% yshift=-40pt] \draw node {\slash 2.\slashAlt 0};%
% \end{tikzpicture}
\begin{tabular}[t]{c|l}%
%\href{http://sistemas.uniandes.edu.co/}%
%     {Departamento de Ingenier�a de Sistemas y Computaci�n} & \\
%\href{http://www.uniandes.edu.co/}{Universidad de los Andes}
\raisebox{-3pt}{\PhoneHandset}
& (313) 200-1812 \\
\raisebox{-3pt}{\Phone} 
& +57(1) 870-8806 \\
\raisebox{-3pt}{\Envelope}
& 
\href{mailto:ericson.cepeda@sukhaweb.com}{ericson.cepeda@sukhaweb.com}
% &	\icons{http://twitter.com/ericsonlopez} {twitter.png} 
% % 	\icons{http://delicious.com/} 		{rss.png}
% 	\icons{http://www.facebook.com/ericson.lopez} 	{facebook.png}
% % 	\icons{http://www.flickr.com/} 		{rss.png} 
% % 	\icons{http://www.last.fm/} 			{rss.png} 
% % 	\icons{http://www.vimeo.com/} 		{rss.png} 
% % 	\icons{http://www.stumbleupon.com/} 	{rss.png}
% % 	\icons{http://www.reddit.com/} 		{rss.png} 
% 	\icons{http://www.linkedin.com/}		{linked.png} 
\end{tabular}
}}%
\end{minipage}

\section{Perfil}
%
Estudiante de pregrado en ingenier\'ia de sistemas con alta motivaci\'on y capaz
de desarrollar aplicaciones de software en diferentes plataformas, llevando a
cabo las etapas de dise\~{n}o correspondientes, en forma cooperativa o
individual. Reconociendo adem\'as los procesos de negocio que pueden influenciar
la implementaci\'on de una soluci\'on.  Esto es complementado con un amplio conocimiento de diversos sistemas operativos.

\section{Educaci\'on}
%
\textbf{Ingenier\'ia de Sistemas y Computaci\'on \hfill{2005 - presente}}
\begin{outerlist}

\item[] \href{http://www.uniandes.edu.co/}{Universidad de los Andes},
Bogot\'a, Cundinamarca Colombia

\end{outerlist}

\section{Experiencia Profesional}
%
\textbf{Desarrollador Web Junior (freelance)} \hfill \textbf{Noviembre 2010 -
presente}
\begin{outerlist}

\item[] \href{http://convertimedia.com/}{Converti Media}, Bogot\'a,
Colombia 
\begin{innerlist}
\item Miner\'ia de datos en bases de datos de Wordpress, y WHMCS para
manejo de \'ordenes, validaci\'on de usuarios, y verificaci\'on de dominios
disponibles en la Web.
\item Programaci\'on en lenguajes como PHP y Javascript (jQuery) de las
p\'aginas web necesarias para la prestaci\'on de un servicio determinado.
\end{innerlist}

\item[] \href{http://irradiadiseno.com/}{Irradia Dise\~no}, Bogot\'a,
Colombia 
\begin{innerlist}
\item Programaci\'on con base en Joomla de distintos sitios web, orientados a
requerimientos espec\'ificos definidos por el cliente, y/o el dise\~nador.
\item Codificaci\'on en lenguajes PHP y Javascript (jQuery) de las
p\'aginas web requeridas para el lanzamiento de un producto.
\item Administraci\'on del DBMS en el servidor del cliente, para adecuar el
comportamiento de las bases de datos, por medio de SQL y/o MySQL, a los
requerimientos funcionales suministrados.
\end{innerlist}

\item[] \href{http://www.sukhaweb.com/}{Sukha SAS.}, Bogot\'a,
Colombia 
\begin{innerlist}
\item Programaci\'on con base en Joomla, y administraci\'on del sitio web de la
empresa.
\item Codificaci\'on en lenguajes PHP, y Javascript (jQuery) de los
requerimientos funcionales, y no funcionales definidos por la organizaci\'on.
\item Administraci\'on del DBMS en el servidor del cliente, y de la
informaci\'on contenida en las bases de datos correspondientes al negocio.
\end{innerlist}
\end{outerlist}

\halfblankline 

\textbf{Desarrollador de sistemas} \hfill \textbf{Marzo 2011 -
presente}
\begin{outerlist}

\item[] \href{http://www.piedradigital.com/}{Piedra Digital}, Bogot\'a,
Colombia
\begin{innerlist}
\item Integrante del grupo de desarrollo de los proyectos Aurora(CakePHP) y
Mediros2(Grails)
\item Programaci\'on con base en Grails, y Java, de sistemas orientados
a requerimientos funcionales espec\'ificos.
\item Integraci\'on de Javascript (jQuery) y Grails, para crear p\'aginas con
contenido din\'amico.
\end{innerlist}

\end{outerlist}

\pagebreak 

\section{Portafolio}
%
Proyectos en curso o terminados

\begin{outerlist}
\item[SUKHA SAS] \href{http://www.sukhaweb.com/}{sukhaweb.com}
\item[A LA CARTA]
\href{http://www.alacartagourmet.com/}{alacartagourmet.com}
% \item[] \thumbnail	{http://www.voiceoverplace.com/}
% 				{voiceov.png}
% 		\thumbnail	{http://www.alacartagourmet.com/final/}
% 				{alacarta.jpg}
% 		\thumbnail	{http://www.sukhaweb.com/}
% 				{sukha}				
\end{outerlist}

\section{Habilidades t\'ecnicas}
%
Experiencia en hardware y software para redes, y tecnolog\'ia de la
informaci\'on

\halfblankline

Programaci\'on: C, C$+$$+$, Java (JUnit, JSF, Servlet, JSP, J2EE), Groovy
(Grails), JavaScript (jQuery), PHP (CakePHP), SQL (MySQL,
PostgreSQL, SQLServer), Python (UNIX shell scripting), Express-MP, Matlab, VHDL,
Git y otros

\halfblankline

Pruebas de software: Documentaci\'on necesaria para registrar las pruebas a
realizar sobre un proyecto en desarrollo. Incluyendo formatos para
documentaci\'on de errores y resultados de pruebas. Lo anterior, con est\'andares de la IEEE

\halfblankline

Tecnolog\'ia de la informaci\'on: Redes (UDP, TCP, ARP, DNS, Dynamic
        routing), Servicio (Apache, Wiki, Wamp Server, phpMyAdmin,
        cPanel, Wordpress, Joomla, dotProject, Planning Tool)

\halfblankline

Aplicaciones de computador: \LaTeX{}, Eclipse, SSH Secure Shell, BiZZdesign
Architect, paquetes comunes de productividad (para Windows, y plataformas Linux)
        , Adobe: Illustrator, Flash Catalyst, Flash Builder, y Dreamweaver

\halfblankline

Dise\~{n}o asistido: Xilinx, SPICE

\halfblankline

Sistemas operativos: Familia Microsoft Windows, Ubuntu, y otras variantes UNIX

\section{Idiomas}
%
Espa\~{n}ol: Nativo. Ingl\'es: Avanzado.

\halfblankline

Ingl\'es: IELTS: 6.0 (Diciembre 2009) Lectura: Excelente. Escritura: Muy buena.
Escucha y habla: Buena

\halfblankline

International House Bristol (The Language Project), Bristol, UK - Curso: General English 20, 1 de Febrero - 7 de Mayo, 2010.
% \newpage

\section{Intereses y logros}
%
\textbf{Investigaci\'on individual}

\begin{outerlist}

\item[] Agent-based modeling, hybrid systems, distributed algorithms, cloud computing, artificial intelligence%
\end{outerlist}

\halfblankline

\textbf{Curso de lectura integral\hfill Enero - Mayo 2005}
\begin{outerlist}

\item[] \href{http://www.tecnicasamericanas.com/}{\textit{T\'ecnicas Americanas
de Estudio}}, Bogot\'a, Cundinamarca Colombia%
\begin{innerlist}
\item Mejoramiento de comprensi\'on y velocidad de lectura
\end{innerlist}
\end{outerlist}

\halfblankline

\textbf{Servicio voluntario\hfill 25 de Mayo - 5 de Junio 2010}
\begin{outerlist}

\item[] \href{http://www.dipa.dhamma.org/}{Dhamma Dipa Meditation Centre},
Hereford, UK%
\begin{innerlist}
\item Miembro del equipo de cocina para 150 estudiantes de meditacion, durante
10 d\'ias
\end{innerlist}
\end{outerlist}

\section{Referencias}
%
Disponibles en caso de ser requeridas.
% \href{mailto:mfonseca@procalculo.com}{\textbf{M\'onica Fonseca Puentes}}
% \begin{outerlist}
% 
% \item[] Directora Help Desk%
%         \hfill \href{http://www.procalculoprosis.com/}{\textbf{Procalculo Prosis S.A.}}
% \begin{innerlist}
% \item Tel: +57(1) 6501550 Ext.3313    
% \item Cel: (300) 5704573  
% \item Cel: (310) 7534086
% \end{innerlist}
% \end{outerlist}
% 
% \halfblankline
% 
% \href{mailto:samafi85@gmail.com}{\textbf{Sandra Madrigal Fierro}}
% \begin{outerlist}
% 
% \item[] Ingeniera en ICBF (Servicio al consumidor)%
%         \hfill \href{https://www.icbf.gov.co/}{\textbf{ICBF}}
% \begin{innerlist}
% \item Cel: (316) 636-2777    
% \end{innerlist}
% \end{outerlist}
% \end{spanish}
% \begin{english}
% \makeheading{Ericson Cepeda}

\section{Contact Information}
%
% NOTE: Mind where the & separators and \\ breaks are in the following
%       table.
%
% ALSO: \rcollength is the width of the right column of the table
%       (adjust it to your liking; default is 1.85in).
%
\begin{minipage}[t]{\textwidth-\rcollength-0.5cm}
Rinc\'{o}n de Santa Ana\newline
Cra 7 Este \# 25 - 59\newline
Ch\'{i}a, Cundinamarca Colombia
\end{minipage}
\begin{minipage}[t]{\rcollength-0.5cm}
\colorbox{shade}{\textcolor{text1}{
% \begin{tikzpicture}[overlay, opacity=0.8, color=black, xshift=0.3cm,
% yshift=-40pt] \draw node {\slash 2.\slashAlt 0};%
% \end{tikzpicture}
\begin{tabular}[t]{@{}c|l}%
%\href{http://sistemas.uniandes.edu.co/}%
%     {Departamento de Ingenier�a de Sistemas y Computaci�n} & \\
%\href{http://www.uniandes.edu.co/}{Universidad de los Andes}
\raisebox{-3pt}{\PhoneHandset} 
& (313) 200-1812 \\
\raisebox{-3pt}{\Phone} 
& +57(1) 870-8806 \\
\raisebox{-3pt}{\Envelope}
& 
\href{mailto:e-cepeda@uniandes.edu.co}{e-cepeda@uniandes.edu.co}\\
% &	\icons{http://twitter.com/ericsonlopez} {twitter.png}  
% 	\icons{http://delicious.com/} 		{rss.png}
% 	\icons{http://www.facebook.com/ericson.lopez} 	{facebook.png}
% 	\icons{http://www.flickr.com/} 		{rss.png} 
% 	\icons{http://www.last.fm/} 			{rss.png} 
% 	\icons{http://www.vimeo.com/} 		{rss.png} 
% 	\icons{http://www.stumbleupon.com/} 	{rss.png}
% 	\icons{http://www.reddit.com/} 		{rss.png} 
% 	\icons{http://www.linkedin.com/}		{linked.png} 
\end{tabular}
}}%
\end{minipage}


\section{Profile}
%
Undergraduate student of Computer Science, highly motivated, and in capacity of developing software applications on different platforms, implementing different stages in group or individually. Besides, recognising the business processes involved in the implementation of a solution. This is complemented with a wide knowledge of different operative systems.

%\section{Security Clearance}
%
%Department of Defense Top Secret SCI with polygraph (expired: 2002)

% \section{Citizenship}
%
% Colombia

\section{Education}
%
Computer Science, 2005-present
\begin{outerlist}

\item[] \href{http://www.uniandes.edu.co/}{\textbf{Universidad de los Andes}},
Bogot\'{a}, Colombia

\end{outerlist}

\section{Technical Skills}
%
% Hardware and software experience in networking, and information technology
% 
% \halfblankline
% 
Programming: C, C$+$$+$, Java (JUnit, JSF, Servlet, JSP, J2EE), Groovy
(Grails), JavaScript (jQuery), PHP (CakePHP), SQL (MySQL,
PostgreSQL), Python (UNIX shell scripting), Express-MP, Matlab, VHDL, and others

\halfblankline

Information Technology: Networking (UDP, TCP, ARP, DNS, Dynamic
        routing), Service (Apache, Wiki, Wamp Server, phpMyAdmin,
        cPanel, Wordpress, Joomla, dotProject, Planning Tool)

\halfblankline

Computer Applications: \LaTeX{}, Eclipse, SSH Secure Shell, BiZZdesign
Architect, most common productivity packages (for Windows, and Linux platforms)
        , Adobe: Illustrator, Flash Catalyst, Flash Builder, Dreamweaver, and others

\halfblankline

Computer-Aided Design: Xilinx, SPICE

\halfblankline

Operative Systems: Microsoft Windows family,  Ubuntu, and other UNIX variants

\section{Languages}
%
Spanish: Native. English: Advanced

\halfblankline

English: IELTS: 6.0 (December 2009) Reading: Excellent. Writing: Very good. Listening and speaking: Good

\halfblankline

International House Bristol (The Language Project), Bristol, UK - Curso: General English 20, 1st February - 7th May, 2010

\section{Interests and Achivements}
%
\textbf{Individual research}
\begin{outerlist}
\item[] Agent-based modeling, hybrid systems, distributed algorithms, cloud computing, artificial intelligence%
\end{outerlist}

\halfblankline

% \textbf{Non-academic activities}
% 
% \begin{outerlist}
% 
% \item[] I frequently practise sports, such as basketball or swimming. Altough, I spend some time searching information about science advances and technology, as well as, improving my intellect doing some reading on literature, or being autodidact with some informatics thematics%
% \end{outerlist}
% 
% \blankline

\textbf{Reading course}
\begin{outerlist}

\item[] \href{http://www.tecnicasamericanas.com/}{\textit{T\'ecnicas Americanas
de Estudio}}, Bogot\'a, Colombia%
        \\ \textbf{January - May 2005}
\begin{innerlist}
\item Reading comprenhension, and speed improving
\end{innerlist}
\end{outerlist}

\halfblankline

\textbf{Voluntary service}
\begin{outerlist}

\item[] \href{http://www.dipa.dhamma.org/}{\textit{Dhamma Dipa Meditation Centre}},
Hereford, UK%
        \\ \textbf{25th May - 5th June 2010}
\begin{innerlist}
\item Member of the kitchen team for 150 meditation students, during 10 days
\end{innerlist}
\end{outerlist}

\section{References}
%
Available upon request
%\href{mailto:mfonseca@procalculo.com}{\textbf{M�nica Fonseca Puentes}}
%\begin{outerlist}
%
%\item[] \textit{Directora Help Desk}%
%        \hfill \href{http://www.procalculoprosis.com/}{\textbf{Procalculo Prosis S.A.}}
%\begin{innerlist}
%\item \textit{Tel:} +57(1) 6501550 Ext.3313    
%\item \textit{Cel:} (300) 5704573  
%\item \textit{Cel:} (310) 7534086
%\end{innerlist}
%\end{outerlist}
%
%\blankline
%
%\href{mailto:samafi85@gmail.com}{\textbf{Sandra Madrigal Fierro}}
%\begin{outerlist}
%
%\item[] \textit{Ingeniera en ICBF (Servicio al consumidor)}%
%        \hfill \href{https://www.icbf.gov.co/}{\textbf{ICBF}}
%\begin{innerlist}
%\item \textit{Cel:} (316) 636-2777    
%\end{innerlist}
%\end{outerlist}
% \end{english}

\end{document}

%%%%%%%%%%%%%%%%%%%%%%%%%% End CV Document %%%%%%%%%%%%%%%%%%%%%%%%%%%%%