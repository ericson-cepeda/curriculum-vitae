
%%%%%%%%%%%%%%%%%%%%%%%%%%%%%%%%%%%%%%%%%%%%%%%%%%%%%%%%%%%%%%%%%%%%%%%%
%%%%%%%%%%%%%%%%%%%%%% Simple LaTeX CV Template %%%%%%%%%%%%%%%%%%%%%%%%
%%%%%%%%%%%%%%%%%%%%%%%%%%%%%%%%%%%%%%%%%%%%%%%%%%%%%%%%%%%%%%%%%%%%%%%%

%%%%%%%%%%%%%%%%%%%%%%%%%%%%%%%%%%%%%%%%%%%%%%%%%%%%%%%%%%%%%%%%%%%%%%%%
%% NOTE: If you find that it says                                     %%
%%                                                                    %%
%%                           1 of ??                                  %%
%%                                                                    %%
%% at the bottom of your first page, this means that the AUX file     %%
%% was not available when you ran LaTeX on this source. Simply RERUN  %%
%% LaTeX to get the ``??'' replaced with the number of the last page  %%
%% of the document. The AUX file will be generated on the first run   %%
%% of LaTeX and used on the second run to fill in all of the          %%
%% references.                                                        %%
%%%%%%%%%%%%%%%%%%%%%%%%%%%%%%%%%%%%%%%%%%%%%%%%%%%%%%%%%%%%%%%%%%%%%%%%

%%%%%%%%%%%%%%%%%%%%%%%%%%%% Document Setup %%%%%%%%%%%%%%%%%%%%%%%%%%%%

% Don't like 10pt? Try 11pt or 12pt
\documentclass[11pt
%							,spanish
%							,draft
							]{article}

\usepackage{calc}
%symbols - the ones you see on the left of the email and of the phone
\usepackage{bbding} 
% \usepackage{marvosym} 
%Colors/Graphics
\usepackage{color,graphicx}
\usepackage[table,usenames,dvipsnames]{xcolor}
\usepackage[pscoord]{eso-pic}% The zero point of the coordinate systemis the lower left corner of the page (the default).
\usepackage{tikz}
\usetikzlibrary{backgrounds}
\usepackage{background}
\usepackage{wrapfig}

\usepackage{fontawesome}
% \usepackage{boxedminipage}
% This is a helpful package that puts math inside length specifications
\usepackage{fontspec}
\usepackage{xunicode}
\usepackage{xltxtra}
\usepackage{polyglossia}
\usepackage{setspace}
\setdefaultlanguage{spanish}
\setotherlanguages{english}
% \usepackage[british,spanish]{babel}
% \usepackage[latin1]{inputenc}
% \usepackage[T1]{fontenc} 	% Better PDF on screen visualisation (a)
% \usepackage{ae,aecompl}
% \usepackage{titlesec}
% Simpler bibsection for CV sections
% (thanks to natbib for inspiration)
\makeatletter
\newlength{\bibhang}
\setlength{\bibhang}{1em}
\newlength{\bibsep}
 {\@listi \global\bibsep\itemsep \global\advance\bibsep by\parsep}
\newenvironment{bibsection}%
        {\vspace{-\baselineskip}\begin{list}{}{%
       \setlength{\leftmargin}{\bibhang}%
       \setlength{\itemindent}{-\leftmargin}%
       \setlength{\itemsep}{\bibsep}%
       \setlength{\parsep}{\z@}%
        \setlength{\partopsep}{0cm}%
        \setlength{\topsep}{0cm}}}
        {\end{list}\vspace{-0.6\baselineskip}}
\makeatother

% Layout: Puts the section titles on left side of page
\reversemarginpar

%
%         PAPER SIZE, PAGE NUMBER, AND DOCUMENT LAYOUT NOTES:
%
% The next \usepackage line changes the layout for CV style section
% headings as marginal notes. It also sets up the paper size as either
% letter or A4. By default, letter was used. If A4 paper is desired,
% comment out the letterpaper lines and uncomment the a4paper lines.
%
% As you can see, the margin widths and section title widths can be
% easily adjusted.
%
% ALSO: Notice that the includefoot option can be commented OUT in order
% to put the PAGE NUMBER *IN* the bottom margin. This will make the
% effective text area larger.
%
% IF YOU WISH TO REMOVE THE ``of LASTPAGE'' next to each page number,
% see the note about the +LP and -LP lines below. Comment out the +LP
% and uncomment the -LP.
%
% IF YOU WISH TO REMOVE PAGE NUMBERS, be sure that the includefoot line 
% is uncommented and ALSO uncomment the \pagestyle{empty} a few lines
% below.
%

%% Use these lines for letter-sized paper
\usepackage[paper=letterpaper,
            %includefoot, % Uncomment to put page number above margin
            marginparwidth=0.92in,     % Length of section titles
            marginparsep=0.5in,       % Space between titles and text
            margin=1in,               % 1 inch margins
            includemp]{geometry}
 
%% Use these lines for A4-sized paper
%\usepackage[paper=a4paper,
%            %includefoot, % Uncomment to put page number above margin
%            marginparwidth=30.5mm,    % Length of section titles
%            marginparsep=1.5mm,       % Space between titles and text
%            margin=25mm,              % 25mm margins
%            includemp]{geometry}

%% More layout: Get rid of indenting throughout entire document
\setlength{\parindent}{0in}
% \setlength{\unitlength}{1in}
%% This gives us fun enumeration environments. compactitem will be nice.
\usepackage{paralist}

%% Reference the last page in the page number
%
% NOTE: comment the +LP line and uncomment the -LP line to have page
%       numbers without the ``of ##'' last page reference)
%
% NOTE: uncomment the \pagestyle{empty} line to get rid of all page
%       numbers (make sure includefoot is commented out above)
%
\usepackage{fancyhdr,lastpage}
\pagestyle{fancy}
%\pagestyle{empty}      % Uncomment this to get rid of page numbers
\fancyhf{}\renewcommand{\headrulewidth}{0pt}
\fancyfootoffset{\marginparsep + \marginparwidth}
\newlength{\footpageshift}
\setlength{\footpageshift}{0.5\textwidth + 0.5\marginparsep +
0.5\marginparwidth-2in} \lfoot{\hspace{\footpageshift}%
       \parbox{4in}{\, \hfill %
                    \arabic{page} of \protect\pageref*{LastPage} % +LP
%                    \arabic{page}                               % -LP
                    \hfill \,}}
% \fancyhf[HL]{\setlength{\unitlength}{1in}
% \begin{tikzpicture}[remember picture, overlay]
% \draw [style=background, xshift=-6.5cm, yshift=2.5cm, very thin] (0,-290mm)
% rectangle (20mm, 0mm);
% \end{tikzpicture}}
% Finally, give us PDF bookmarks
\usepackage{color,hyperref}
\definecolor{darkblue}{rgb}{0.0,0.0,0.3}%{HTML}{F4D455}
\definecolor{darkgreen}{HTML}{066769}
\definecolor{namebox}{HTML}{45B6BC}

\definecolor{stripe}{HTML}{E4F4F4}
\definecolor{shade}{HTML}{D4D7FE}	%light blue shade
\definecolor{text1}{HTML}{2b2b2b}		%text is almost black
\definecolor{text2}{HTML}{2D2D2D}		%text is almost black
\definecolor{textitem}{HTML}{5DC8CE}		%text is almost black
\definecolor{subtitle}{HTML}{A1DFE3}		%text is almost black
\definecolor{backgroundColor}{rgb}{0.0,0.0,0.3}%{HTML}{183090}% {HTML}{F4D455}
%%%%%%%%%%%%%%%%%%%%%%%% End Document Setup %%%%%%%%%%%%%%%%%%%%%%%%%%%%
 

%%%%%%%%%%%%%%%%%%%%%%%%%%% Helper Commands %%%%%%%%%%%%%%%%%%%%%%%%%%%%
%Fonts and Tweaks for XeLaTeX


\setmainfont[Color=text2]{Calibri}
\font\headers="Calibri:letterspace=5" at 20pt
\font\SectionHeaders="Calibri:letterspace=5" at 14pt
%\font\headers="Qlassik Bold:letterspace=5" at 22pt
%\font\SectionHeaders="Qlassik Bold:letterspace=5" at 14pt
\font\Career="Calibri:color=2b2b2b" at 13pt 
\font\Text="Calibri:color=2D2D2D" at 11pt 
\font\TextAlt="Calibri:color=CC3300" at 11pt
\font\TextSC="Bebas Neue:+smcp, color=066769, letterspace=4" at 16pt 
\font\slash="Bebas Neue:+zero, color=2b2b2b, letterspace=4" at 34.5pt
\font\slashWhite="Bebas Neue:+zero, color=fffffe, letterspace=4" at 34.5pt
\font\slashGray="Bebas Neue:+zero, color=808080, letterspace=4" at 34.5pt
\font\slashAlt="Corbel:+zero, color=CC3300" at 15pt
\font\ContactHeaders="Corbel" at 170 pt
\font\trick="Corbel:color=FFFFFF" at 0.1 pt
\newfontfamily\FA[Color=darkgreen]{FontAwesome Regular}

\hypersetup{colorlinks,breaklinks,
            linkcolor=darkgreen,urlcolor=darkgreen,
            anchorcolor=darkgreen,citecolor=darkgreen,filecolor=darkgreen}
\makeatletter
\def\HyColor@@@@UseColor#1\@nil{\addfontfeatures{Color=#1}}
\makeatother

\tikzset{background/.style={fill=backgroundColor}}
\tikzset{background grid/.style = {thick, draw  = blue, step = .5cm}}
%Thumbnail for the Portfolio
\setlength\fboxsep{0.1mm}
\newcommand{% 
	\thumbnail}[2]{
		\raggedright{\href{#1}
			{\trick \raisebox{-1pt}{ }
% 				\fbox{
			\includegraphics[keepaspectratio=true,height=20pt]{#2}}
			\raisebox{20pt}{ }}}
% 			}
%Social ICONS 
\newcommand{\icons}[2]{
		{\href{#1}
		{\trick \raisebox{-1pt}{.}
			\raisebox{-4pt}{
			\includegraphics[width=10pt,height=10pt]{#2}}
		\raisebox{10pt}{.}}}
		}
% The title (name) with a horizontal rule under it
%
% Usage: \makeheading{name}
%
% Place at top of document. It should be the first thing.
\newcommand{\makeheading}[2]%%
        {\hspace*{-\marginparsep minus \marginparwidth}%
         \begin{minipage}[l]{\textwidth + \marginparwidth
         +\marginparsep} {\large\slash\setlength{\fboxsep}{2.2mm}
                \hspace*{-2\fboxsep}\colorbox{namebox}{\slashWhite #1}
                \slashGray #2}%
         \end{minipage}}
         
 \newcommand{\placetextbox}[3]{% \placetextbox{<horizontal pos>}{<vertical pos>}{<stuff>}
  \setbox0=\hbox{#3}% Put <stuff> in a box
  \AddToShipoutPictureFG*{% Add <stuff> to current page foreground
    \put(\LenToUnit{#1\paperwidth},\LenToUnit{#2\paperheight}){\vtop{{\null}\makebox[0pt][c]{#3}}}%
  }%
}%
% The section headings
%
% Usage: \section{section name}
%
% Follow this section IMMEDIATELY with the first line of the section
% text. Do not put whitespace in between. That is, do this:
%
%       \section{My Information}
%       Here is my information.
%
% and NOT this:
%
%       \section{My Information}
%
%       Here is my information.
%
% Otherwise the top of the section header will not line up with the top
% of the section. Of course, using a single comment character (%) on
% empty lines allows for the function of the first example with the
% readability of the second example.
\renewcommand{\section}[2]%
        {\pagebreak[2]\vspace{1.3\baselineskip}%
         \phantomsection\addcontentsline{toc}{section}{#1}%
         \hspace{0in}%
         \marginpar{ 
         \vspace{-8pt}\raggedright\setstretch{1.4}\TextSC#1}\Text#2}

% An itemize-style list with lots of space between items
\newenvironment{outerlist}[1][\textbullet]%
        {\begin{itemize}[#1]}{\end{itemize}%
         \vspace{-.6\baselineskip}}

% An environment IDENTICAL to outerlist that has better pre-list spacing
% when used as the first thing in a \section
\newenvironment{lonelist}[1][\enskip\textbullet]%
        {\vspace{-\baselineskip}\begin{list}{#1}{%
        \setlength{\partopsep}{0pt}%
        \setlength{\topsep}{0pt}}}
        {\end{list}\vspace{-.6\baselineskip}}

% An itemize-style list with little space between items
\newenvironment{innerlist}[1][\textbullet]%
        {\begin{compactitem}[#1]}{\end{compactitem}}

% An environment IDENTICAL to innerlist that has better pre-list spacing
% when used as the first thing in a \section
\newenvironment{loneinnerlist}[1][\textbullet]%
        {\vspace{-\baselineskip}\begin{compactitem}[#1]}
        {\end{compactitem}\vspace{-.6\baselineskip}}

% To add some paragraph space between lines.
% This also tells LaTeX to preferably break a page on one of these gaps
% if there is a needed pagebreak nearby.
\newcommand{\blankline}{\quad\pagebreak[2]}
\newcommand{\halfblankline}{\quad\vspace{-0.5\baselineskip}\pagebreak[3]}
\newcommand{\quarterblankline}{\quad\vspace{0.1\baselineskip}\pagebreak[3]}
\newcommand{\smallblankline}{\vspace{0.3\baselineskip}\pagebreak[3]}

% Uses hyperref to link DOI
\newcommand\doilink[1]{\href{http://dx.doi.org/#1}{#1}}
\newcommand\doi[1]{doi:\doilink{#1}}
 
\newlength{\rcollength}\setlength{\rcollength}{2.3in}%

\SetBgScale{1}
\SetBgAngle{0}
\SetBgColor{stripe}
\SetBgContents{\rule{1.5in}{\paperheight}}
\SetBgHshift{-.5\paperwidth+1.577in}

%%%%%%%%%%%%%%%%%%%%%%%% End Helper Commands %%%%%%%%%%%%%%%%%%%%%%%%%%%

%%%%%%%%%%%%%%%%%%%%%%%%% Begin CV Document %%%%%%%%%%%%%%%%%%%%%%%%%%%%

\includeonly{espanol, british}

\begin{document} 

% \begin{spanish}
% \makeheading{Ericson Cepeda}

\section{Informaci\'on personal}
%
% NOTE: Mind where the & separators and \\ breaks are in the following
%       table.
%
% ALSO: \rcollength is the width of the right column of the table
%       (adjust it to your liking; default is 1.85in).
%
\begin{minipage}[t]{\textwidth-\rcollength-0.5cm}
Rinc\'{o}n de Santa Ana\newline
Cra 7 Este \# 25 - 59\newline
Ch\'{i}a, Cundinamarca Colombia
\end{minipage}
\begin{minipage}[t]{\rcollength-0.5cm}
\colorbox{shade}{\textcolor{text1}{ 
% \begin{tikzpicture}[overlay, opacity=0.8, color=black, xshift=0.3cm,
% yshift=-40pt] \draw node {\slash 2.\slashAlt 0};%
% \end{tikzpicture}
\begin{tabular}[t]{c|l}%
%\href{http://sistemas.uniandes.edu.co/}%
%     {Departamento de Ingenier�a de Sistemas y Computaci�n} & \\
%\href{http://www.uniandes.edu.co/}{Universidad de los Andes}
\raisebox{-3pt}{\PhoneHandset} 
& $+57\:3005574311$ \\
\raisebox{-3pt}{\Phone} 
& $+57\:8708806$ \\
\raisebox{-3pt}{\Envelope}
& 
\href{mailto:e-cepeda@uniandes.edu.co}{e-cepeda@uniandes.edu.co}\\
% &	\icons{http://twitter.com/ericsonlopez} {twitter.png} 
% % 	\icons{http://delicious.com/} 		{rss.png}
% 	\icons{http://www.facebook.com/ericson.lopez} 	{facebook.png}
% % 	\icons{http://www.flickr.com/} 		{rss.png} 
% % 	\icons{http://www.last.fm/} 			{rss.png} 
% % 	\icons{http://www.vimeo.com/} 		{rss.png} 
% % 	\icons{http://www.stumbleupon.com/} 	{rss.png}
% % 	\icons{http://www.reddit.com/} 		{rss.png} 
% 	\icons{http://www.linkedin.com/}		{linked.png} 
\end{tabular}
}}%
\end{minipage}

\section{Perfil}
%
Estudiante de pregrado en ingenier\'ia de sistemas con alta motivaci\'on y capaz
de desarrollar aplicaciones de software en diferentes plataformas, llevando a
cabo las etapas de dise\~{n}o correspondientes, en forma cooperativa o
individual. Reconociendo adem\'as los procesos de negocio que pueden influenciar
la implementaci\'on de una soluci\'on.  Esto es complementado con un amplio conocimiento de diversos sistemas operativos.

\section{Educaci\'on}
%
\href{http://www.uniandes.edu.co/}{\textbf{Universidad de los Andes}},
Bogot\'{a}, Colombia
\begin{outerlist}
\item[] Ingenier\'ia de Sistemas y Computaci\'on \hfill \textbf{2005 - Presente}
%     \begin{innerlist}
%     	\item GPA: $3.5$
%     \end{innerlist}
\end{outerlist}

\section{Experiencia Profesional}
%
\href{http://convertimedia.com/}{\textbf{Converti Media}}, Bogot\'a, Colombia.
Empresa de consultor\'ia y desarrollo de estrategias de comunicacion en l\'inea.
\begin{outerlist}
\item[] \textbf{Desarrollador Junior} \hfill \textbf{Noviembre 2010 - Mayo 2011}
    \begin{innerlist}
    	\item Dise\~no e implementaci\'on de algoritmos necesarios para la
    	integraci\'on de las funcionalidades de WHMCS con una aplica\'on online
    	Wordpress.
		\item Miner\'ia de datos (MySQL), Desarrollo en WHMCS y Wordpress.
		\item PHP, Javascript (jQuery) y servicios web.
    \end{innerlist}
\end{outerlist}

\quarterblankline

\href{http://irradiadiseno.com/}{\textbf{Irradia Dise\~no}}, Bogot\'a, Colombia.
Grupos de dise\~nadores con enfoque en las necesidades de las PYMEs en Colombia.
\begin{outerlist}
\item[] \textbf{Desarrollador Web (freelance)} \hfill \textbf{Junio 2011 - Julio
2011}
    \begin{innerlist}
\item Reconstrucci\'on de las funcionalidades b\'asicas de
\href{http://www.alacartagourmet.com/}{\textbf{A La Carta}}, habilitando el 100\% de los clientes para
hacer \'ordenes en l\'inea.
\item PHP (Joomla) y Javascript (jQuery).
\item Administraci\'on de DBMS MySQL.
    \end{innerlist}
\end{outerlist}

\quarterblankline

\href{http://www.sukhaweb.com/}{\textbf{Sukha SAS.}}, Bogot\'a, Colombia.
Compa\~n\'ia multiprop\'osito que integra: Comercio internacional de productos,
desarrollo software, automatizaci\'on de hogares y comercio de productos
org\'anicos.
\begin{outerlist}
\item[] \textbf{Administrador TI} \hfill \textbf{Febrero 2011 - Present}
    \begin{innerlist}
\item Construcci\'on del sitio web principal de la compa\~n\'ia, en busca de
formas creativas y minimalistas para mostrar el perfil de la empresa.
Administraci\'on de la infrestructura tecnol\'ogica.
\item PHP (CakePHP) y Javascript (jQuery).
\item Administraci\'on DBMS MySQL.
    \end{innerlist}
\end{outerlist}

\quarterblankline

\href{http://www.piedradigital.com/}{\textbf{Piedra Digital}}, Bogot\'a,
Colombia. Compa\~n\'ia de desarrollo software con enfoque en el equilibrio entre
soluciones estrat\'egicas y su implementaci\'on.
\begin{outerlist}
\item[] \textbf{Systems analyst and developer} \hfill \textbf{Marzo 2011 - Junio
2012}
    \begin{innerlist}
\item Conclusi\'on de la producci\'on y lanzamiento de dos proyectos: Aurora
(CakePHP) y
\href{http://demo.100grados.co:8080/desempeno100/}{\textbf{Mediros2}} (Grails).
\item Java (core, Grails) y Javascript (jQuery).
\item Miner\'ia de datos con SQL Server y PostgreSQL.
    \end{innerlist}
\end{outerlist}

\quarterblankline

\href{http://sqbluesky.com/}{\textbf{SQBlueSky Inc.}}, Dallas, TX USA. 
Compa\~n\'ia financiera enfocada en ofrecer y construir herramientas \'utiles
para la investigaci\'on de inversi\'on.
\begin{outerlist}
\item[] \textbf{Systems analyst and developer} \hfill \textbf{Mayo 2012 -
Octubre 2012}
    \begin{innerlist}
\item Integraci\'on de la aplicaci\'on web de la empresa:
\href{http://app.insideredge.com/}{\textbf{Insider Edge}}
y la nueva implementaci\'on CouchDB, para un servicio r\'apido y m\'as
confiable.
\item PHP (Zend Framework) y Javascript (jQuery).
\item Miner\'ia de datos con CouchDB y MySQL.
    \end{innerlist}
\end{outerlist}

\quarterblankline

\href{http://www.enteract.com.au/}{\textbf{Enteract}}, Melbourne, Australia.
Empresa l\'ider en mercadeo. Produce un amplio rango de servicios de desarrollo
web, mercadeo online y eventos de comercio.
\begin{outerlist}
\item[] \textbf{Web developer} \hfill \textbf{Julio 2012 - Febrero 2013}
    \begin{innerlist}
\item Coordinaci\'on de la actualizaci\'on del sitio para distribuci\'on de
tajetas virtuales: 
\href{https://my.ekarda.com/}{\textbf{Ekarda}} hacia tecnolog\'ias m\'as
recientes, incrementando la compatibilidad con nuevos requerimientos.
\item PHP (CakePHP) y Javascript (jQuery).
\item An\'alisis de c\'odigo con REGEX avanzado para la implementaci\'on de
i18n en reemplazo de texto plano.
    \end{innerlist}
\end{outerlist}

% \pagebreak 

\section{Portafolio}
%
Proyectos:

\begin{innerlist}
\item \href{http://www.alacartagourmet.com/}{\textbf{A La Carta}}
\item \href{http://www.sukhaweb.com/}{\textbf{Sukha SAS}}
\item \href{http://demo.100grados.co:8080/desempeno100/}{\textbf{Mediros2}}
\item \href{http://app.insideredge.com/}{\textbf{Insider Edge}}
\item \href{https://my.ekarda.com/}{\textbf{Ekarda}}
\end{innerlist}
% \begin{outerlist}
% \item[SUKHA SAS] \href{http://www.sukhaweb.com/}{sukhaweb.com}
% \item[A LA CARTA]
% \href{http://www.alacartagourmet.com/}{alacartagourmet.com}
% \item[] \thumbnail	{http://www.voiceoverplace.com/}
% 				{voiceov.png}
% 		\thumbnail	{http://www.alacartagourmet.com/final/}
% 				{alacarta.jpg}
% 		\thumbnail	{http://www.sukhaweb.com/}
% 				{sukha}				
% \end{outerlist}

\section{Habilidades t\'ecnicas}
%
\textbf{Programaci\'on:}

    \begin{innerlist}
\item Avanzado: Core Java 
\item Intermedio: Groovy (Grails), JavaScript (jQuery), PHP (CakePHP, Zend,
Wordpress, Joomla), SQL (MySQL, PostgreSQL, SQL Server), NoSQL (MongoDB, CouchDB), Python (Django, Gevent, Selenium), Ruby on Rails, \LaTeX{},  Java (JUnit, JSF, Servlet, JSP,
J2EE, JEE)
\item Aprendiz: C, C$+$$+$, C\#, Express-MP, Matlab, VHDL, Hadoop, Python (UNIX
shell scripting).
\item Principiante: Objective C
    \end{innerlist}

\halfblankline

\textbf{Tecnolog\'ia de la informaci\'on:} 
    \begin{innerlist}
\item Networking (UDP, TCP, ARP, DNS, Dynamic
        routing)
\item Servicio (Apache, Wiki, Wamp Server, phpMyAdmin,
        cPanel)
    \end{innerlist}

\halfblankline

\textbf{Aplicaciones:} 
    \begin{innerlist}
\item Eclipse, SSH Secure Shell, BiZZdesign
Architect, Adobe: Illustrator, Flash Catalyst, Flash Builder, Dreamweaver, and others
\item Most common productivity packages (for Windows, and Linux platforms)
    \end{innerlist}

\halfblankline

\textbf{Dise\~no asistido:} 
    \begin{innerlist}
\item Xilinx, SPICE
    \end{innerlist}


\halfblankline

\textbf{Sistemas operativos:}
    \begin{innerlist}
\item Familia Microsoft Windows  
\item Ubuntu y otras distribuciones UNIX
    \end{innerlist}

\section{Idiomas}
%
Espa\~{n}ol: Nativo. Ingl\'es: Avanzado.

\halfblankline

Ingl\'es: IELTS: 6.0 (Diciembre 2009) Lectura: Excelente. Escritura: Muy buena.
Escucha y habla: Buena

\halfblankline

International House Bristol (The Language Project), Bristol, UK - Curso: General English 20, 1 de Febrero - 7 de Mayo, 2010.
% \newpage

\section{Intereses y logros}
%
\textbf{Individual}

\begin{outerlist}
\item[] Ha crecido un sentimiento de humanidad y responsabilidad social
asociado a algunas experiencias con la meditaci\'on. Uno de los objetivos es
ayudar con a las personas con amor incondicional y al mundo a ser un mejor
lugar para vivir. Existen lugares donde se puede encontrar la motivaci\'on
para lograr dicho objectivo desde la tecnolog\'ia, por lo que existe ese impulso
para ser tecnol\'ogicamente ``verde''.
%
\end{outerlist}

\halfblankline

\textbf{Curso de lectura integral\hfill Enero - Mayo 2005}
\begin{outerlist}

\item[] \href{http://www.tecnicasamericanas.com/}{\textit{T\'ecnicas Americanas
de Estudio}}, Bogot\'a, Cundinamarca Colombia%
\begin{innerlist}
\item Mejoramiento de comprensi\'on y velocidad de lectura
\end{innerlist}
\end{outerlist}

\halfblankline

\textbf{Servicio voluntario\hfill 25 de Mayo - 5 de Junio 2010}
\begin{outerlist}

\item[] \href{http://www.dipa.dhamma.org/}{Dhamma Dipa Meditation Centre},
Hereford, UK%
\begin{innerlist}
\item Miembro del equipo de cocina para 150 estudiantes de meditacion, durante
10 d\'ias
\end{innerlist}
\end{outerlist}

\section{Referencias}
%
Disponibles en caso de ser requeridas.
% \href{mailto:mfonseca@procalculo.com}{\textbf{M\'onica Fonseca Puentes}}
% \begin{outerlist}
% 
% \item[] Directora Help Desk%
%         \hfill \href{http://www.procalculoprosis.com/}{\textbf{Procalculo Prosis S.A.}}
% \begin{innerlist}
% \item Tel: +57(1) 6501550 Ext.3313    
% \item Cel: (300) 5704573  
% \item Cel: (310) 7534086
% \end{innerlist}
% \end{outerlist}
% 
% \halfblankline
% 
% \href{mailto:samafi85@gmail.com}{\textbf{Sandra Madrigal Fierro}}
% \begin{outerlist}
% 
% \item[] Ingeniera en ICBF (Servicio al consumidor)%
%         \hfill \href{https://www.icbf.gov.co/}{\textbf{ICBF}}
% \begin{innerlist}
% \item Cel: (316) 636-2777    
% \end{innerlist}
% \end{outerlist}
% \end{spanish}
\begin{english}
\makeheading{Ericson}{Cepeda}{
\placetextbox{0.75}{.97}{
%
		\begin{tikzpicture}[remember picture,overlay]
		\draw[dashed, color=namebox, thin, double, double distance=0pt] (0pt,.5cm)
		-- (0pt,-3.4cm);
		\end{tikzpicture}
		\hspace*{0pt}\makebox[6.5cm][c]{%
		\def\arraystretch{1.1}%
		\begin{tabular}[t]{cl}%
%\href{http://sistemas.uniandes.edu.co/}%
%     {Departamento de Ingenier\'ia de Sistemas y Computaci\'on} & \\
%\href{http://www.uniandes.edu.co/}{Universidad de los Andes}
% \raisebox{-1pt}{\FA \faMobile} 
% & \small +57 3005574311 \\
\raisebox{-1pt}{\FA \faGlobe}
& 
\href{http://www.picorb.com}{picorb.com} \\
\raisebox{-1pt}{\FA \faEnvelopeAlt}
& 
\href{mailto:ericson@picorb.com}{ericson@picorb.com} \\
\raisebox{-1pt}{\FA \faLinkedinSign}
& 
\href{https://co.linkedin.com/in/ericsonlopez}{ericsonlopez} \\
\raisebox{-1pt}{\FA \faGithub}
& 
\href{https://github.com/ericson-cepeda}{ericson-cepeda} \\
\raisebox{-1pt}{\FA \faBitbucket}
& 
\href{https://bitbucket.org/ericson_cepeda}{ericson\_cepeda} \\
\raisebox{-1pt}{\FA \faPhoneSign} 
& \small +1 8576000106 \\
\raisebox{-1pt}{\FA \faMapMarker}
& 
\small Cra 7 Este 25-59, Chía CO-CUN
% &	\icons{http://twitter.com/ericsonlopez} {twitter.png} 
% % 	\icons{http://delicious.com/} 		{rss.png}
% 	\icons{http://www.facebook.com/ericson.lopez} 	{facebook.png}
% % 	\icons{http://www.flickr.com/} 		{rss.png} 
% % 	\icons{http://www.last.fm/} 			{rss.png} 
% % 	\icons{http://www.vimeo.com/} 		{rss.png} 
% % 	\icons{http://www.stumbleupon.com/} 	{rss.png}
% % 	\icons{http://www.reddit.com/} 		{rss.png} 
% 	\icons{http://www.linkedin.com/}		{linked.png} 
\end{tabular}
}}%
}%
         
         

         
\vspace{0.7cm}
% \section{Contact Information}
%
% NOTE: Mind where the & separators and \\ breaks are in the following
%       table.
%
% ALSO: \rcollength is the width of the right column of the table
%       (adjust it to your liking; default is 1.85in).
%
% \begin{minipage}[t]{\textwidth-\rcollength-0.5cm}
% Rinc\'{o}n de Santa Ana\newline
% Cra $7$ Este $\# 25-59$\newline
% Ch\'{i}a, Cundinamarca Colombia
% \end{minipage}
% \begin{minipage}[t]{\rcollength-0.5cm}
% \colorbox{shade}{\textcolor{text1}{
% \begin{tikzpicture}[overlay, opacity=0.8, color=black, xshift=0.3cm,
% yshift=-40pt] \draw node {\slash 2.\slashAlt 0};%
% \end{tikzpicture}
% \begin{tabular}[t]{c|l@{}}%
%\href{http://sistemas.uniandes.edu.co/}%
%     {Departamento de Ingeniería de Sistemas y Computación} & \\
%\href{http://www.uniandes.edu.co/}{Universidad de los Andes}
% \raisebox{-3pt}{\PhoneHandset} 
% & $+57\:3005574311$ \\
% \raisebox{-3pt}{\Phone} 
% & $+57\:18708806$ \\
% \raisebox{-3pt}{\Envelope}
% & 
% \href{mailto:hello.ericson@picorb.com}{hello.ericson@picorb.com}\\
% &	\icons{http://twitter.com/ericsonlopez} {twitter.png}  
% 	\icons{http://delicious.com/} 		{rss.png}
% 	\icons{http://www.facebook.com/ericson.lopez} 	{facebook.png}
% 	\icons{http://www.flickr.com/} 		{rss.png} 
% 	\icons{http://www.last.fm/} 			{rss.png} 
% 	\icons{http://www.vimeo.com/} 		{rss.png} 
% 	\icons{http://www.stumbleupon.com/} 	{rss.png}
% 	\icons{http://www.reddit.com/} 		{rss.png} 
% 	\icons{http://www.linkedin.com/}		{linked.png} 
% \end{tabular}
% }}%
% \end{minipage}


\section{Profile}
%\\\smallText{Spanish: Native\\English: Advanced}
%Full stack computer systems engineer, highly motivated and in capacity
%of implementing software applications on different platforms, following
%different stages in group or individually, making use of agile methodologies.
% Besides, recognising the business processes involved in the implementation of a solution. This is complemented with a wide knowledge of different operative systems.
Full stack computer systems engineer, curious about what comes next and how to
take advantage to build a better place to coexist and enjoy.
Self-taught person willing to share
the learnings acquired and
conscious of the uncanny ability to learn in humans. Eager for knowledge when
mentors or mixed mindsets come around.
Aware of this agile and fast-paced world where we live, knowing that there exist
processes and rules, but also chaos which brings new dimensions to reality.

\vspace{3.5mm}
\textbf{Spanish} Native. \textbf{English} Advanced.
%\section{Security Clearance}
%
%Department of Defense Top Secret SCI with polygraph (expired: 2002)

% \section{Citizenship}
%
% Colombia

\section{Education}
%
% \href{http://idic.edu.co/}{\textbf{IDIC}},
% Bogot\'{a}, Colombia
% \begin{outerlist}
% \item[] Grades: Kindergarten - $11$th \hfill \textbf{1992 - 2004}
% \end{outerlist}
% 
% \halfblankline 
% 
\href{http://www.topuniversities.com/node/2823/ranking-details/latin-american-university-rankings/2014}{\textbf{Universidad de los Andes}},
Bogot\'{a}, CO-CUN
\begin{outerlist}
\item[\FA \faAngleDoubleRight] \textbf{BS in Systems and Computing Engineering}
\hfill \textbf{2014}
%     \begin{innerlist}
%     	\item GPA: $3.5$
%     \end{innerlist}
\end{outerlist}

\section{Professional Experience}
%
\href{http://www.twnel.com/}{\textbf{Twnel Inc.}}, \textit{Boston, US-MA.
Startup.
Pioneer in mobile messaging for business.}

\begin{outerlist}
\item[\FA \faAngleDoubleRight] \textbf{Infrastructure and Software Engineer}
\hfill
\textbf{Nov 2014 - Present}
\end{outerlist}

\begin{innerlist}
\item Engineered a new distributed infrastructure going from
unpredictable to +99.7\% uptime per service, improving resources
utilization up to 25\% and simplifying interaction with core services for 100\%
of team members.
\item Reduced operational and infrastructure costs up to 50\%, using automated
deployment procedures (beta, production, test harness).
\item \textbf{Technology:} AWS, CoreOS, Docker, NodeJS, ReactJS, Python,
Ansible, CircleCI.
\end{innerlist}

\quarterblankline

\href{http://alertlogic.com/}{\textbf{Alert Logic}}, \textit{Houston, US-TX.
Pioneer in cloud solutions that are secure, flexible and designed to work with
hosting and cloud service providers.}

\begin{outerlist}
\item[\FA \faAngleDoubleRight] \textbf{Software Development Engineer (SDE,
SDET)}
\hfill
\textbf{Mar 2013 - Present}
\end{outerlist}

    \begin{innerlist}
\item Developed new features for WSM project (health checking, packet
filtering).
\item Built
successful solutions through automation in white and black box testing types and
levels to integrate tools used by 4 teams, setting a stand point for quality
assurance and reducing the time spent to define tests documentation by 65\%.
\item Agile Software Development (SCRUM), Test Driven Development.
\item \textbf{Technology:} PHP (Selenium Framework), Python, Perl, Erlang, Ruby,
AWS.
    \end{innerlist}

\quarterblankline


% \href{http://convertimedia.com/}{\textbf{Converti Media}}, Bogot\'a, Colombia.
% Consultancy and online communication strategies development enterprise,
% specialized on digital media.
% \begin{outerlist}
% \item[] \textbf{Junior web developer} \hfill \textbf{November 2010 - May 2011}
%     \begin{innerlist}
%     	\item Engineered implementations to bring the
%     	WHMCS functionalities, from the account held by the company, into a
%     	brand new Wordpress site.
% 		\item Data mining (MySQL), Wordpress developing and WHMCS management for
% 		domain reselling.
% 		\item PHP and Javascript (jQuery) programming for specified web services.
%     \end{innerlist}
% \end{outerlist}

% \quarterblankline

% \href{http://irradiadiseno.com/}{\textbf{Irradia Dise\~no}}, Bogot\'a, Colombia.
% Young designers group with an effective and creative approach for the needs of
% SMEs in Colombia.
% \begin{outerlist}
% \item[] \textbf{Web developer (freelance)} \hfill \textbf{June 2011 - July 2011}
%     \begin{innerlist}
% \item Researched and redesigned the core functionalities of
% \textbf{A La Carta}, in such
% way it would behave as required, so 100\% of the clients were able to order
% online.
% \item PHP (Joomla) and Javascript (jQuery) programming of required web pages.
% \item MySQL DBMS management to meet new requirements.
%     \end{innerlist}
% \end{outerlist}
% 
% \quarterblankline
%

% \href{http://sqbluesky.com/}{\textbf{SQBlueSky Inc.}}, Dallas, TX USA. Small
% niche financial company focused on building and offering useful investment research tools.
% \begin{outerlist}
% \item[] \textbf{Systems analyst and developer} \hfill \textbf{May 2012 - October
% 2012}
%     \begin{innerlist}
% \item Guided the integration between \textbf{Insider Edge} 
% website and the new CouchDB implementation, for a faster and more reliable
% service.
% \item PHP (Zend Framework) and Javascript (jQuery) based programming.
% \item Data mining with CouchDB and MySQL.
%     \end{innerlist}
% \end{outerlist}
% 
% \quarterblankline

% \href{http://www.enteract.com.au/}{\textbf{Enteract}}, Melbourne, Australia.
% Leading digital marketing company. They deliver
% a broad range of web development, email marketing \& trade event services.
% \begin{outerlist}
% \item[] \textbf{Web developer (freelance)} \hfill \textbf{July 2012 - February
% 2013}
%     \begin{innerlist}
% \item Achieved the upgrade of the product for business e-cards distribution:
% \href{https://my.ekarda.com/}{\textbf{Ekarda}} to recent web technologies,
% increasing the compatibility with new plugins and features.
% \item PHP (CakePHP) and Javascript (jQuery) based programming.
% \item Code analysis with advanced REGEX to implement i18n rather than plain
% text.
%     \end{innerlist}
% \end{outerlist}

% \quarterblankline

\href{http://www.picorb.com/}{\textbf{PicOrb}}, \textit{Ch\'ia, CO-CUN.
Digital agency. Freelancing from 2010 and formalized in
2013.}

\begin{outerlist}
\item[\FA \faAngleDoubleRight] \textbf{CIO (founder)} \hfill \textbf{Jan 2013 -
Present}
\end{outerlist}

    \begin{innerlist}
\item Engineered the company's official website.
\item Lead and collaborate with a dynamic team in several
web projects, aiming to performance, scalability and ease of use with
the best practices.
\item Integrated cloud services for continuous integration and
deployment.
\item \textbf{Technology:} Django, AngularJS, AWS, Heroku, Codeship.
    \end{innerlist}

\quarterblankline

\textbf{Piedra Digital},
\textit{Bogot\'a, CO-CUN. Consultancy and software development
company, which gives balance between strategic solutions and their implementation.}

\begin{outerlist}
\item[\FA \faAngleDoubleRight] \textbf{Systems Analyst and Developer} \hfill
\textbf{March 2011 - June 2012}
\end{outerlist}

    \begin{innerlist}
\item Accomplished the production and release of
two projects in a rapid development team: Aurora and
\href{http://demo.100grados.co:8080/desempeno100/}{\textbf{Mediros2}}.
\item Integrated backend libraries for data mining.
\item \textbf{Technology:} CakePHP, jQuery, SQL Server, PostgreSQL, Grails.
    \end{innerlist}

\quarterblankline

\textbf{Sukha SAS.}, \textit{Bogot\'a, CO-CUN.
Multipurpose company:
international product trading, software development, home-automation and organic
products fair-trading.}

\begin{outerlist}
\item[\FA \faAngleDoubleRight] \textbf{Lead Software Engineer (co-founder)}
\hfill \textbf{February 2011 - October 2012}
\end{outerlist}

    \begin{innerlist}
\item Built the company main website seeking for creative,
minimalistic and attractive ways to show the company profile.
\item \textbf{Technology:} CakePHP, JQuery, MySQL DBMS, CPanel.
    \end{innerlist}
% \blankline




% \pagebreak 

% \section{Portfolio}
%
% Freelance:

% \begin{innerlist}
% \item \href{http://www.picorb.com/}{\textbf{PicOrb}}
% \item \href{http://www.cafetosoftware.com/}{\textbf{Cafeto Software}}
% \item \href{http://www.giant-turkey.com/}{\textbf{Giant Turkey}}
% \item \href{http://www.allbikers.net/}{\textbf{All Bikers}}
% \item \href{https://my.ekarda.com/}{\textbf{Ekarda}}
% \item \href{http://demo.100grados.co:8080/desempeno100/}{\textbf{Mediros2}}
% \item \href{http://www.sukhaweb.com/}{\textbf{Sukha SAS}}
% \item \href{http://www.alacartagourmet.com/}{\textbf{A La Carta}}
% \end{innerlist}
% 
% \begin{innerlist}
% \item[] \thumbnail	{http://www.voiceoverplace.com/}{mediros.png}
% 		\thumbnail	{http://www.alacartagourmet.com/final/}{carta.jpg}
% 		\thumbnail	{http://www.sukhaweb.com/}{sukha}
% 		\thumbnail	{http://www.sukhaweb.com/}{insider.png}
% \end{innerlist}
% \quarterblankline

% Contractual:

% \begin{innerlist}
% \item \href{http://www.picorb.com/}{\textbf{PicOrb}}
% \item \href{http://www.cafetosoftware.com/}{\textbf{Cafeto Software}}
% \item \href{http://www.giant-turkey.com/}{\textbf{Giant Turkey}}
% \item \href{http://www.allbikers.net/}{\textbf{All Bikers}}
% \item \href{https://my.ekarda.com/}{\textbf{Ekarda}}
% \item \href{http://demo.100grados.co:8080/desempeno100/}{\textbf{Mediros2}}
% \item \href{http://www.sukhaweb.com/}{\textbf{Sukha SAS}}
% \item \href{http://www.alacartagourmet.com/}{\textbf{A La Carta}}
% \end{innerlist}

% \newpage

\section{Technical Skills}
%
% Hardware and software experience in networking, and information technology
% 
% \halfblankline
% 
\textbf{Computer programming:}

    \begin{innerlist}
\item Advanced: Core Java, Python (Django, Selenium), JavaScript
(NodeJS, AngularJS).
\item Intermediate: Groovy (Grails), PHP
(CakePHP, Zend, Wordpress, Joomla), Perl, Ruby (RoR), SQL (MySQL, PostgreSQL,
SQL Server), NoSQL (MongoDB, CouchDB, Neo4j), 
Java (JUnit, J2EE), bash (UNIX shell scripting), \LaTeX{}.
\item Apprentice: C$+$$+$, C\#, Matlab, VHDL, Hadoop, Clojure,
Erlang, Go.
\item Beginner: Objective C, Express-MP.
    \end{innerlist}

\halfblankline

\textbf{Operative systems:}
    \begin{innerlist}
\item Microsoft Windows family.
\item Ubuntu, CoreOS and other UNIX variants.
    \end{innerlist}

\halfblankline

\textbf{Information technology:} 
    \begin{innerlist}
\item Networking (Bind, SkyDNS, Consul, Swarm).
\item Service (NodeJS, Tornado, Nginx, Apache, Unicorn).
\item Continuous integration (Jenkins, Codeship, CircleCI,
Fabric, Ansible, Chef).
\item Cloud computing (Heroku, AWS, Rackspace, Google Cloud Platform).
\item Containerization and virtualization (Docker, Vagrant).
    \end{innerlist}

% \halfblankline
% 
% \textbf{Computer applications:} 
%     \begin{innerlist}
% \item Eclipse, JetBrains suites (PyCharm, IntelliJ, WebStorm, RubyMine), SSH
% Secure Shell, BiZZdesign Architect, Adobe: Illustrator, Flash Catalyst, Flash
% Builder, Dreamweaver, and others.
% \item Most common productivity packages (for Windows, and Linux platforms).
%     \end{innerlist}
% 
% \halfblankline
% 
% \textbf{Computer-aided design:} 
%     \begin{innerlist}
% \item Xilinx, SPICE.
%     \end{innerlist}


% \section{Languages}
%
% Spanish: Native. English: Advanced.
% 
% \halfblankline
% 
% English: Reading: Excellent. Writing: Very good.
% Listening and speaking: Very good.

% \halfblankline
% 
% International House Bristol (The Language Project), Bristol, UK - Course:
% General English 20, 1st February - 7th May, 2010.

\section{Interests and \\ Achivements}
%
\textbf{Reading course}

\begin{outerlist}
\item[\FA \faAngleDoubleRight]
\href{http://www.tecnicasamericanas.com/}{\textbf{T\'ecnicas Americanas de Estudio}}, Bogot\'a, Colombia%
        \hfill \textbf{January - May 2005}
\end{outerlist}

\begin{innerlist}
\item Improved reading comprenhension and speed.
\end{innerlist}

\halfblankline

\textbf{Voluntary service}

\begin{outerlist}
\item[\FA \faAngleDoubleRight] \href{http://www.dipa.dhamma.org/}{\textbf{Dhamma
Dipa Meditation Centre}}, Hereford, UK%
        \hfill \textbf{May - June 2010}
\end{outerlist}

\begin{innerlist}
\item Performed as a member of the kitchen team for 150 Vipassana meditation
students, during 10 days.
\end{innerlist}

% \textbf{Individual}
% \begin{outerlist}
% \item[] Agent-based modeling, hybrid systems, distributed algorithms, cloud computing, artificial intelligence%
% \item[] It has grown a sense of humanity and social
% responsibilitiy  thanks to meditation practise. The main
% goal is helping people with unconditional love, and the world to be a better
% place to live. There are few places where one can meet the needed aim and
% expertise to achieve such objective from technology. That is why there is
% an impulse to be technologically green.
% \end{outerlist}
% 
% \halfblankline

% \textbf{Non-academic activities}
% 
% \begin{outerlist}
% 
% \item[] I frequently practise sports, such as basketball or swimming. Altough, I spend some time searching information about science advances and technology, as well as, improving my intellect doing some reading on literature, or being autodidact with some informatics thematics%
% \end{outerlist}
% 
% \blankline

\section{References}
%
Available upon request.
%\href{mailto:mfonseca@procalculo.com}{\textbf{Mónica Fonseca Puentes}}
%\begin{outerlist}
%
%\item[] Directora Help Desk%
%        \hfill \href{http://www.procalculoprosis.com/}{\textbf{Procalculo Prosis S.A.}}
%\begin{innerlist}
%\item Tel: +57(1) 6501550 Ext.3313    
%\item Cel: (300) 5704573  
%\item Cel: (310) 7534086
%\end{innerlist}
%\end{outerlist}
%
%\blankline
%
%\href{mailto:samafi85@gmail.com}{\textbf{Sandra Madrigal Fierro}}
%\begin{outerlist}
%
%\item[] Ingeniera en ICBF (Servicio al consumidor)%
%        \hfill \href{https://www.icbf.gov.co/}{\textbf{ICBF}}
%\begin{innerlist}
%\item Cel: (316) 636-2777    
%\end{innerlist}
%\end{outerlist}
\end{english}

\end{document}

%%%%%%%%%%%%%%%%%%%%%%%%%% End CV Document %%%%%%%%%%%%%%%%%%%%%%%%%%%%%